\documentclass[11pt]{article}
\usepackage[utf8]{inputenc}
\usepackage[slovene]{babel}

\usepackage{amsthm}
\usepackage{amsmath, amssymb, amsfonts}

\theoremstyle{definition}
\newtheorem{definicija}{Definicija}[section]

\newtheorem{lema}{Lema}
\newtheorem{trditev}{Trditev}
\newtheorem{izrek}{Izrek}
\newtheorem*{dokaz}{Dokaz}

\title{Analiza 3 - definicije, trditve in izreki}
\author{Oskar Vavtar}
\date{2020/21}

\begin{document}
\maketitle
\pagebreak
\tableofcontents
\pagebreak

\section{PARAMETRIČNO PODANE KRIVULJE}

\vspace{0.5cm}	

\begin{trditev}

Če je $\vec{r}$ \textit{odvedljiva} vektorska funkcija (njene komponente $x$, $y$ in $z$ so odvedljive funkcije spremenljivke $t$), potem je
$$\dot{\vec{r}}(t_0) = (\dot{x}(t_0), \dot{y}(t_0), \dot{z}(t_0))$$
\textit{tangentni vektor} na krivuljo $t \mapsto \vec{r}(t)$ v točki $\vec{r}(t_0)$, če velja $\dot{\vec{r}}(t_0) \neq 0$.
	
\end{trditev}

\vspace{0.5cm}

\begin{trditev}

Če je $\vec{r}$ \textit{zvezno odvedljiva} vektorska funkcija na intervalu $[a, b]$ (za	$a < b$), je potem \textit{dolžina} krivulje, ki jo določa, enaka
$$s = \int_{a}^{b} \| \dot{\vec{r}}(t) \| dt.$$
To velja tudi za funkcijo, ki so le \textit{odsekoma zvezne}. Opazimo tudi, da je zgornja dolžina neodvisna od parametrizacije krivulje.	

\end{trditev}

\vspace{0.5cm}

\begin{trditev}

Naj bo $\vec{r}$ \textit{zvezno odvedljiva} vektorska funkcija, definirana na intervalu $[a, b]$ (za $a < b$) in naj bo $\psi: [a, b] \rightarrow [\alpha, \beta]$ \textit{zvezno odvedljiva bijekcija}, tako da $t = \psi(\tau)$ preteče interval $[a, b]$, ko $\tau$ preteče interval $[\alpha, \beta]$ (za $\alpha < \beta)$. Potem je
$$\int_{a}^{b} \| \dot{\vec{r}}(t) \| dt = \int_{\alpha}^{\beta} \| \frac{d}{d \tau} \vec{r}(\psi(\tau)) \| d\tau. $$

\end{trditev}

\end{document}
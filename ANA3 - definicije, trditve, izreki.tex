\documentclass[11pt]{article}
\usepackage[utf8]{inputenc}
\usepackage[slovene]{babel}

\usepackage{amsthm}
\usepackage{amsmath, amssymb, amsfonts}
\usepackage{relsize}

\theoremstyle{definition}
\newtheorem{definicija}{Definicija}[section]

\theoremstyle{definition}
\newtheorem{trditev}{Trditev}[section]

\theoremstyle{definition}
\newtheorem{izrek}{Izrek}[section]

\renewcommand{\qedsymbol}{$\blacksquare$}

\newtheorem{lema}{Lema}
\newtheorem*{posledica}{Posledica}

\title{Analiza 3 - definicije, trditve in izreki}
\author{Oskar Vavtar}
\date{2020/21}

\begin{document}
\maketitle
\pagebreak
\tableofcontents
\pagebreak

\section{PARAMETRIČNO PODANE KRIVULJE}
\vspace{0.5cm}	

\begin{trditev}

Če je $\vec{r}$ \textit{odvedljiva} vektorska funkcija (njene komponente $x$, $y$ in $z$ so odvedljive funkcije spremenljivke $t$), potem je
$$\dot{\vec{r}}(t_0) = (\dot{x}(t_0), \dot{y}(t_0), \dot{z}(t_0))$$
\textit{tangentni vektor} na krivuljo $t \mapsto \vec{r}(t)$ v točki $\vec{r}(t_0)$, če velja $\dot{\vec{r}}(t_0) \neq 0$.
	
\end{trditev}
\vspace{0.5cm}

\begin{trditev}

Če je $\vec{r}$ \textit{zvezno odvedljiva} vektorska funkcija na intervalu $[a, b]$ (za	$a < b$), je potem \textit{dolžina} krivulje, ki jo določa, enaka
$$s = \int_{a}^{b} \| \dot{\vec{r}}(t) \| dt.$$
To velja tudi za funkcijo, ki so le \textit{odsekoma zvezne}. Opazimo tudi, da je zgornja dolžina neodvisna od parametrizacije krivulje.	

\end{trditev}
\vspace{0.5cm}

\begin{trditev}

Naj bo $\vec{r}$ \textit{zvezno odvedljiva} vektorska funkcija, definirana na intervalu $[a, b]$ (za $a < b$) in naj bo $\psi: [a, b] \rightarrow [\alpha, \beta]$ \textit{zvezno odvedljiva bijekcija}, tako da $t = \psi(\tau)$ preteče interval $[a, b]$, ko $\tau$ preteče interval $[\alpha, \beta]$ (za $\alpha < \beta)$. Potem je
$$\int_{a}^{b} \| \dot{\vec{r}}(t) \| dt = \int_{\alpha}^{\beta} \| \frac{d}{d \tau} \vec{r}(\psi(\tau)) \| d\tau. $$

\end{trditev}
\vspace{0.5cm}

% *************************************************************************************************

\pagebreak

% #################################################################################################

\section{PLOSKVE}
\vspace{0.5cm}

% *************************************************************************************************

\subsection{Ploskve v $\mathbb{R}^3$}
\vspace{0.5cm}

\begin{definicija}[Ploskev]

Podmnožica $P \subseteq \mathbb{R}^3$ je \textit{ploskev}, če za vsako točko $\vec{r} \in P$ obstaja taka okolica $H \subseteq \mathbb{R}^3$, da je $P \cap H$ graf kake zvezno odvedljive funkcije $\phi: D \rightarrow \mathbb{R}$, definirane na kaki \textit{odprti} podmnožici $D \subseteq \mathbb{R}^2$. \\
To pomeni, da se na $P \cap H$ ena od koordinat $x, y, z$ da \textit{enolično} izraziti kot funkcija preostalih, torej da je $P \cap H$ ene od oblik:
$$P \cap H = \{(x, y, \phi(x, y)) \mid (x, y) \in D\},$$
$$P \cap H = \{(x, \phi(x, z), y) \mid (x, z) \in D\},$$
$$P \cap H = \{(\phi(y, z), y, z) \mid (y, z) \in D\}.$$

\end{definicija}
\vspace{0.5cm}

\begin{trditev}[Izrek o implicitni funkciji]

Naj bo $g: \mathbb{R}^3 \rightarrow \mathbb{R}$ \textit{zvezno odvedljiva} funkcija in privzemimo, da je množica $P = g^{-1}(0)$ \textit{neprazna}. Če je
$$\nabla g(\vec{r}) \neq 0$$
za $\forall \vec{r} \in P$ je $P$ \textit{ploskev}. \\

Enačba oblike $\vec{r} = \vec{r}(t)$ ($t \in [a, b] \subseteq \mathbb{R}, ~a < b$) predstavlja krivuljo v $\mathbb{R}^3$. Privzeli bomo, da je pri tem $\vec{r}$ \textit{zvezno odvedljiva} funkcija spremenljivke $t$. Taka krivulja leži na ploskvi $P = g^{-1}(0)$ natanko tedaj, ko je $g(\vec{r}(t)) = 0$ za $\forall t \in [a, b]$. Ko to enakost odvajamo po $t$, dobimo
$$\nabla g(\vec{r}(t)) \cdot \dot{\vec{r}}(t) = 0.$$
Ta enakost pomeni, da je vektor $\nabla g(\vec{r}(t))$ pravokoten na tangentni vektor $\dot{\vec{r}}(t)$ krivulje v točki $\vec{r}(t)$. \\

Če sedaj izberemo poljubno točko $\vec{r}_0$ na ploskvi $P$ in opazujemo vse krivulje na ploskvi $P$, ki gredo skozi točko $\vec{r}_0$ (vsaka taka krivulja $\vec{r} = \vec{r}(t)$ zadošča pogoju $\vec{r}(t_0) = \vec{r}_0$ za kak $t_0$), vidimo, da je vektor $\nabla g(\vec{r}_0)$ pravokoten na tangentni vektor $\dot{\vec{r}}(t_0)$ vsake take krivulje. \\

To pomeni, da mora biti vektor $\nabla g(\vec{r}_0)$ pravokoten na ploskev $P$. To velja za vsako točko $\vec{r}_0 \in P$.
 

\end{trditev}
\vspace{0.5cm}

\begin{definicija}[Normalni vektor]

Vektor $\nabla g(\vec{r})$ imenujemo \textit{normalni vektor} na ploskev $P = g^{-1}(0)$ v točki $\vec{r} \in P$. Ravnino $T_{\vec{r}}P$ z normalnim vektorjem $\nabla g(\vec{r})$ skozi točko $\vec{r}$ na ploskvi $P$ pa imenujemo \textit{tangentna ravnina} na ploskev $P$ v točki $\vec{r}$. \\

Tangentna ravnina na $P$ skozi točko $\vec{r}$ je torej vzporedna vsem tangentnim vektorjem v točki $\vec{r}$ na krivulje skozi $\vec{r}$ na ploskvi $P$.

\end{definicija}
\vspace{0.5cm}

% *************************************************************************************************

\pagebreak

% #################################################################################################

\section{INTEGRALI S PARAMETROM}
\vspace{0.5cm}

% *************************************************************************************************

\begin{definicija}[Integral s parametrom]

Naj bo $f$ \textit{zvezna funkcija} dveh spremenljivk, definirana na pravokotniku $P = [a, b] \times [c, d]$ ($a < b$, $c < d$). Integral
\begin{equation}
\label{eq:1}
	F(y) = \int_{a}^{b} f(x, y) dx
\end{equation}

je funkcija spremenljivke $y$. Tak integral imenujemo \textit{integral s parametrom} $y$.

\end{definicija}
\vspace{0.5cm}

\begin{trditev}

Če je $f$ \textit{zvezna funkcija} na pravokotniku $P = [a, b] \times [c, d]$, je funkcija $F$ (definirana z (\ref{eq:1})) \textit{zvezna} na intervalu $P$.

\end{trditev}
\vspace{0.5cm}

\begin{izrek}

Naj bo $f$ \textit{zvezna} na pravokotniku $P = [a, b] \times [c, d]$ in privzemimo, da obstaja parcialni odvod $\cfrac{\partial f}{\partial y}$, ki naj bo \textit{zvezen} na $P$. Potem je funkcija $F$ (podana z (\ref{eq:1})) \textit{odvedljiva} in velja 
\begin{equation}
	F'(y) = \frac{d}{dy} \int_{a}^{b} f(x, y) dx = \int_{a}^{b} \frac{\partial f}{\partial y} (x, y) dx.
\end{equation}

\end{izrek}
\vspace{0.5cm}

% *************************************************************************************************

\subsection{Izlimitirani integrali s parametrom}
\vspace{0.5cm}

\begin{definicija}

Integral $F(y) = \mathlarger{\int_{a}^{\infty} f(x, y) dx}$ je \textit{enakomerno konvergenten} za $y \in S \subseteq \mathbb{R}$, če za $\forall \varepsilon > 0 ~\exists M \in \mathbb{R}$, da za $\forall b \geq M$ in $\forall y \in S$ velja
$$\left| \int_{b}^{\infty} f(x, y) dx \right| < \varepsilon.$$ 

Za razliko od navadne konvergence mora tukaj obstajati tak $M$, ki je istočasno ustrezen za $\forall y \in S$, torej je $M = M_{\varepsilon}$ odvisen le od $\varepsilon$, ne pa tudi od $y$. Pri navadni konvergenci bi bil veljalo $M = M_{\varepsilon,y}$.

\end{definicija}
\vspace{0.5cm}

\begin{trditev}

Če je $f$ \textit{zvezna} funkcija na pasu $P = [a, \infty) \times [c, d]$ in integral
$$F(y) = \int_{a}^{\infty} f(x, y) dx$$
\textit{enakomerno konvergenten} za $y \in [c, d]$, je $F$ \textit{zvezna funkcija} na $[c, d]$.

\end{trditev}
\vspace{0.5cm}

% *************************************************************************************************

\subsection{Dvojni in dvakratni integrali}
\vspace{0.5cm}

\begin{definicija}

Naj bo $P = [a, b] \times [c, d]$ in $f: P \rightarrow \mathbb{R}$ funkcija. Delitev $D_{[a, b]}$ intervala $[a, b]$ je določena z zaporedjem točk
$$a = x_0 < x_1 < \ldots < x_m = b.$$
Delitev $D_{[a, b]}$ skupaj s poljubno delitvijo $D_{[c, d]}$ intervala $[c, d]$, določeno z 
$$c = y_0 < y_1 < \ldots < y_n = d,$$
določa neko delitev pravokotnika $P$ na manjpe pravokotnike
$$P_{i,j} = [x_{i-1}, x_i] \times [y_{i-1}, y_i], ~(i = 1, \ldots, m; ~j = 1, \ldots, n).$$

Naj bo 
$$m_{i,j} = \inf_{(x,y) \in P_{i,j}} f(x, y),$$
$$M_{i,j} = \sup_{(x,y) \in P_{i,j}} f(x, y).$$
Z $\Delta_{i,j} p = \Delta_i x \cdot \Delta_j y = (x_i - x_{i-1})(y_j - y_{j-1})$ označimo ploščino pravokotnika $P_{i,j}$. 
Vsoto
$$\underline{S}_D = \sum_{i=1}^{m} \sum_{j=1}^{n} m_{i,j} \Delta_{i,j} p$$
imenujemo \textit{spodnja}, vsoto
$$\overline{S}_D = \sum_{i=1}^{m} \sum_{j=1}^{n} M_{i,j} \Delta_{i,j} p$$
pa \textit{zgornja Riemannova vsota} funkcije $f$ pri delitvi $D$.

\end{definicija}
\vspace{0.5cm}

\begin{lema}

Če je $N$ nadaljevanje delitve $D$ pravokotnika $P$, za spodnje in zgornje Riemannove vsote poljubne omejene funkcije $f: P \rightarrow \mathbb{R}$ velja
$$\underline{S}_N \geq \underline{S}_D ~\text{in}~ \overline{S}_N \leq \overline{S}_D.$$

\end{lema}
\vspace{0.5cm}

\begin{definicija}

Omejena funkcija $f: P \rightarrow \mathbb{R}$ je na pravokotniku $P$ \textit{integrabilna v Riemannovem smislu}, če velja
$$\underline{S} = \overline{S},$$
kjer je $\underline{S}$ supermum njenih \textit{spodnjih}, $\overline{S}$ pa infimum njenih \textit{zgornjih} Riemannovih vsot. Tedaj skupno vrednost $\underline{S} = \overline{S}$ označimo kot 
$$\iint_P f(x, y) dp,$$
kjer pomeni $dp = dxdy$ \textit{ploščinski element}, in jo imenujemo \textit{dvojni integral funkcije $f$ po pravokotniku $P$}.

\end{definicija}
\vspace{0.5cm}

\begin{izrek}

\textit{Zvezna funkcija} $f$ na pravokotniku $P = [a, b] \times [c, d]$ je \textit{integrabilna} in velja
\begin{equation}
\int_{a}^{b} \left( \int_{c}^{d} f(x, y) dy \right) dx = \iint_P f(x, y) dp = \int_{c}^{d} \left( \int_{a}^{b} f(x, y) dx \right) dy.
\end{equation}

Enak zaključek velja tudi za funkcijo $f$, ki ni nujno zvezna, če je $N$ množica njenih točk nezveznosti taka, da jo za $\forall \varepsilon > 0$ lahko pokrijemo s kakim zaporedjem pravokotnikov, katerih vsota ploščin je pod $\varepsilon$. Tedaj pravimo, da ima $N$ mero $0$.

\end{izrek}
\vspace{0.5cm}

\begin{posledica}

Za funkcijo $f$, ki je na pravokotniku $P$ \textit{integrabilna} v Riemannovem smislu, konvergirajo Riemannove vsote $S$ proti $\mathlarger{\iint_P f(x, y) dp}$, ko gredo velikosti delilnih pravokotnikov (njihove diagonale) proti $0$. \\

Natančneje: za $\forall \varepsilon > 0 ~\exists \delta > 0$, da je 
$$\left| S - \iint_P f(x, y) dp \right| < \varepsilon$$
za vsako Riemannovo vsoto funkcije $f$ pri vsaki delitvi pravokotnika $P$, kjer si dolžine diagonal pod $\delta$.

\end{posledica}
\vspace{0.5cm}

% *************************************************************************************************

\subsection{Integriranje in odvajanje integralov s parametrom}
\vspace{0.5cm}

\begin{izrek}

Naj bo $f$ \textit{zvezna} na pasu $[a, \infty) \times [c, d]$. Če je integral $\mathlarger{\int_a^{\infty} f(x, y) dx}$ \textit{enakomerno konvergenten} za $y \in [c, d]$, potem je
$$\int_c^d \int_a^{\infty} f(x, y) dx ~dy = \int_a^{\infty} \int_c^d f(x, y) dy ~dx.$$

\end{izrek}
\vspace{0.5cm}

\begin{izrek}

Naj bosta $f$ in $\cfrac{\partial f}{\partial y}$ \textit{zvezni} na pasu $[a, \infty) \times [c, d]$, naj bo integral
$$F(y) = \int_a^{\infty} f(x, y) dx$$
\textit{konvergenten} za $y \in [c, d]$ in naj bo integral
$$\int_a^{\infty} \frac{\partial f}{\partial y}(x, y) dx$$
\textit{enakomerno konvergenten} na $[c, d]$. Potem je $F$ \textit{odvedljiva} funkcija in velja
$$F'(y) = \frac{d}{dy} \int_a^{\infty} f(x, y) dx = \int_a^{\infty} \frac{\partial f}{\partial y}(x, y) dx.$$

\end{izrek}
\vspace{0.5cm}

\begin{izrek}[Kriterij za ugotavljanje enakomerne konvergence]
~ \\
Integral $\mathlarger{\int_a^{\infty} f(x, y) dx} = F(y)$ je \textit{enakomerno konvergenten} na $S$ natanko tedaj, ko za $\forall \varepsilon > 0 ~\exists N \in \mathbb{R}$, da za poljubna $d > b \geq N$ in za $\forall y \in S$ velja
$$\left| \int_b^d f(x, y) dx \right| < \varepsilon.$$

\end{izrek}
\vspace{0.5cm}

\begin{posledica}

Če je $|f(x, y)| \leq g(x, y)$ za $\forall (x, y) \in [a, \infty) \times [c, d]$ in je integral $\mathlarger{\int_a^b g(x, y) dx}$ \textit{enakomerno konvergenten} na $[c, d]$, je \textit{enakomerno konvergenten} tudi integral $\mathlarger{\int_a^b f(x, y) dx}$.

\end{posledica}
\vspace{0.5cm}

\begin{izrek}[2. izrek o povprečju]

Naj bo $f$ \textit{integrabilna}, $g$ pa \textit{nenegativna padajoča (odvedljiva)} funkcija na intervalu $[a, b]$. Potem $\exists \xi \in [a, b]$, da je
$$\int_a^b f(x) g(x) dx = g(a) \int_a^{\xi} f(x) dx.$$

\end{izrek}
\vspace{0.5cm}

% *************************************************************************************************

\subsection{Eulerjeva funkcija $\Gamma$}
\vspace{0.5cm}

\begin{definicija}[Funkcija $\Gamma$]

Na poltraku $x > 0$ je funkcija $\Gamma$ definirana z
\begin{equation}
	\Gamma(x) = \int_0^{\infty} t^{t-1} e^{-t} dt.
\end{equation}

\end{definicija}
\vspace{0.5cm}

\begin{trditev}[Rekurzivna formula]

Za $\forall x > 0$ velja
$$\Gamma(x + 1) = x \Gamma(x).$$

\end{trditev}
\vspace{0.5cm}

\begin{posledica}

$\Gamma(n + 1) = n!$ za $\forall n \in \mathbb{N}$ \\

\noindent To nam namiguje, naj definiramo
$$x! := \Gamma(x + 1) ~\text{za}~ \forall n \in \mathbb{N}.$$
Rekurzivna formula nam omogoča, da razširimo definicijsko območje funkcije $\Gamma$. Če je namreč $x \in (-1, 0)$, je $x + 1 \in  (0, 1)$, zato je vrednost $\Gamma(x + 1)$ že definiramo in lahko postavimo
$$\Gamma := \frac{\Gamma(x+1)}{x}.$$
S ponavljanjem rekurzivne formule dobimo
\begin{equation} \label{eq:2}
	\Gamma(x) = \frac{\Gamma(x+n)}{x(x+1)\ldots(x+n-1)}.
\end{equation}
Za $\forall x \in \mathbb{R}$, ki ni negativno celo število ali $0$, lahko izberemo tak najmanjši $n \in \mathbb{N}$, da je $(x + n) > 0$; tedaj je vrednost $\Gamma(x + n)$ že definirana in lahko $\Gamma(x)$ definiramo s formulo (\ref{eq:2}).

\end{posledica}
\vspace{0.5cm}

\begin{definicija}

Funkcija beta je definirana kot
\begin{equation} \label{eq:3}
	B(x, y) = \int_0^1 t^{x-1} (1-t)^{y-1} dt, ~~(x>0, y>0).
\end{equation}
Lahko se je prepričati, da je integral v (\ref{eq:3}) konvergenten, če je $x > 0$ in $y > 0$. \\

Z vpeljavo nove integracijske spremenljivke $t = \sin^2{\varphi}$ lahko definicijo funkcije $B$ zapišemo tudi kot
\begin{equation} \label{eq:4}
	B(x, y) = 2 \int_0^{\frac{\pi}{2}} \sin^{2x-1}{\varphi} \cos^{2y-1}{\varphi} d\varphi.
\end{equation}

\end{definicija}
\vspace{0.5cm}

\begin{trditev}

Za poljubna pozitivna $x, y$ je 
\begin{equation} \label{eq:5}
	B(x, y) = \frac{\Gamma(x)\Gamma(y)}{\Gamma(x+y)}
\end{equation}

\end{trditev}
\vspace{0.5cm}

\end{document}
\documentclass[11pt]{article}
\usepackage[utf8]{inputenc}
\usepackage[slovene]{babel}

\usepackage{amsthm}
\usepackage{amsmath, amssymb, amsfonts}
\usepackage{relsize}

\theoremstyle{definition}
\newtheorem{definicija}{Definicija}[section]

\theoremstyle{definition}
\newtheorem{trditev}{Trditev}[section]

\theoremstyle{definition}
\newtheorem{izrek}{Izrek}[section]

\renewcommand{\qedsymbol}{$\blacksquare$}

\newtheorem{lema}{Lema}
\newtheorem*{posledica}{Posledica}

\title{Analiza 3 - definicije, trditve in izreki}
\author{Oskar Vavtar}
\date{2020/21}

\begin{document}
\maketitle
\pagebreak
\tableofcontents
\pagebreak

\section{PARAMETRIČNO PODANE KRIVULJE}
\vspace{0.5cm}	

\begin{trditev}

Če je $\vec{r}$ \textit{odvedljiva} vektorska funkcija (njene komponente $x$, $y$ in $z$ so odvedljive funkcije spremenljivke $t$), potem je
$$\dot{\vec{r}}(t_0) = (\dot{x}(t_0), \dot{y}(t_0), \dot{z}(t_0))$$
\textit{tangentni vektor} na krivuljo $t \mapsto \vec{r}(t)$ v točki $\vec{r}(t_0)$, če velja $\dot{\vec{r}}(t_0) \neq 0$.
	
\end{trditev}
\vspace{0.5cm}

\begin{trditev}

Če je $\vec{r}$ \textit{zvezno odvedljiva} vektorska funkcija na intervalu $[a, b]$ (za	$a < b$), je potem \textit{dolžina} krivulje, ki jo določa, enaka
$$s = \int_{a}^{b} \| \dot{\vec{r}}(t) \| dt.$$
To velja tudi za funkcijo, ki so le \textit{odsekoma zvezne}. Opazimo tudi, da je zgornja dolžina neodvisna od parametrizacije krivulje.	

\end{trditev}
\vspace{0.5cm}

\begin{trditev}

Naj bo $\vec{r}$ \textit{zvezno odvedljiva} vektorska funkcija, definirana na intervalu $[a, b]$ (za $a < b$) in naj bo $\psi: [a, b] \rightarrow [\alpha, \beta]$ \textit{zvezno odvedljiva bijekcija}, tako da $t = \psi(\tau)$ preteče interval $[a, b]$, ko $\tau$ preteče interval $[\alpha, \beta]$ (za $\alpha < \beta)$. Potem je
$$\int_{a}^{b} \| \dot{\vec{r}}(t) \| dt = \int_{\alpha}^{\beta} \| \frac{d}{d \tau} \vec{r}(\psi(\tau)) \| d\tau. $$

\end{trditev}
\vspace{0.5cm}

% *************************************************************************************************

\pagebreak

% #################################################################################################

\section{PLOSKVE V $\mathbb{R}^3$}
\vspace{0.5cm}

% *************************************************************************************************

\begin{definicija}[Ploskev]

Podmnožica $P \subseteq \mathbb{R}^3$ je \textit{ploskev}, če za vsako točko $\vec{r} \in P$ obstaja taka okolica $H \subseteq \mathbb{R}^3$, da je $P \cap H$ graf kake zvezno odvedljive funkcije $\phi: D \rightarrow \mathbb{R}$, definirane na kaki \textit{odprti} podmnožici $D \subseteq \mathbb{R}^2$. \\
To pomeni, da se na $P \cap H$ ena od koordinat $x, y, z$ da \textit{enolično} izraziti kot funkcija preostalih, torej da je $P \cap H$ ene od oblik:
$$P \cap H = \{(x, y, \phi(x, y)) \mid (x, y) \in D\},$$
$$P \cap H = \{(x, \phi(x, z), y) \mid (x, z) \in D\},$$
$$P \cap H = \{(\phi(y, z), y, z) \mid (y, z) \in D\}.$$

\end{definicija}
\vspace{0.5cm}

\begin{trditev}[Izrek o implicitni funkciji]

Naj bo $g: \mathbb{R}^3 \rightarrow \mathbb{R}$ \textit{zvezno odvedljiva} funkcija in privzemimo, da je množica $P = g^{-1}(0)$ \textit{neprazna}. Če je
$$\nabla g(\vec{r}) \neq 0$$
za $\forall \vec{r} \in P$ je $P$ \textit{ploskev}. \\

Enačba oblike $\vec{r} = \vec{r}(t)$ ($t \in [a, b] \subseteq \mathbb{R}, ~a < b$) predstavlja krivuljo v $\mathbb{R}^3$. Privzeli bomo, da je pri tem $\vec{r}$ \textit{zvezno odvedljiva} funkcija spremenljivke $t$. Taka krivulja leži na ploskvi $P = g^{-1}(0)$ natanko tedaj, ko je $g(\vec{r}(t)) = 0$ za $\forall t \in [a, b]$. Ko to enakost odvajamo po $t$, dobimo
$$\nabla g(\vec{r}(t)) \cdot \dot{\vec{r}}(t) = 0.$$
Ta enakost pomeni, da je vektor $\nabla g(\vec{r}(t))$ pravokoten na tangentni vektor $\dot{\vec{r}}(t)$ krivulje v točki $\vec{r}(t)$. \\

Če sedaj izberemo poljubno točko $\vec{r}_0$ na ploskvi $P$ in opazujemo vse krivulje na ploskvi $P$, ki gredo skozi točko $\vec{r}_0$ (vsaka taka krivulja $\vec{r} = \vec{r}(t)$ zadošča pogoju $\vec{r}(t_0) = \vec{r}_0$ za kak $t_0$), vidimo, da je vektor $\nabla g(\vec{r}_0)$ pravokoten na tangentni vektor $\dot{\vec{r}}(t_0)$ vsake take krivulje. \\

To pomeni, da mora biti vektor $\nabla g(\vec{r}_0)$ pravokoten na ploskev $P$. To velja za vsako točko $\vec{r}_0 \in P$.
 

\end{trditev}
\vspace{0.5cm}

\begin{definicija}[Normalni vektor]

Vektor $\nabla g(\vec{r})$ imenujemo \textit{normalni vektor} na ploskev $P = g^{-1}(0)$ v točki $\vec{r} \in P$. Ravnino $T_{\vec{r}}P$ z normalnim vektorjem $\nabla g(\vec{r})$ skozi točko $\vec{r}$ na ploskvi $P$ pa imenujemo \textit{tangentna ravnina} na ploskev $P$ v točki $\vec{r}$. \\

Tangentna ravnina na $P$ skozi točko $\vec{r}$ je torej vzporedna vsem tangentnim vektorjem v točki $\vec{r}$ na krivulje skozi $\vec{r}$ na ploskvi $P$.

\end{definicija}
\vspace{0.5cm}

% *************************************************************************************************

\pagebreak

% #################################################################################################

\section{INTEGRALI S PARAMETROM}
\vspace{0.5cm}

% *************************************************************************************************

\begin{definicija}[Integral s parametrom]

Naj bo $f$ \textit{zvezna funkcija} dveh spremenljivk, definirana na pravokotniku $P = [a, b] \times [c, d]$ ($a < b$, $c < d$). Integral
\begin{equation}
\label{eq:1}
	F(y) = \int_{a}^{b} f(x, y) dx
\end{equation}

je funkcija spremenljivke $y$. Tak integral imenujemo \textit{integral s parametrom} $y$.

\end{definicija}
\vspace{0.5cm}

\begin{trditev}

Če je $f$ \textit{zvezna funkcija} na pravokotniku $P = [a, b] \times [c, d]$, je funkcija $F$ (definirana z (\ref{eq:1})) \textit{zvezna} na intervalu $P$.

\end{trditev}
\vspace{0.5cm}

\begin{izrek}

Naj bo $f$ \textit{zvezna} na pravokotniku $P = [a, b] \times [c, d]$ in privzemimo, da obstaja parcialni odvod $\cfrac{\partial f}{\partial y}$, ki naj bo \textit{zvezen} na $P$. Potem je funkcija $F$ (podana z (\ref{eq:1})) \textit{odvedljiva} in velja 
\begin{equation}
	F'(y) = \frac{d}{dy} \int_{a}^{b} f(x, y) dx = \int_{a}^{b} \frac{\partial f}{\partial y} (x, y) dx.
\end{equation}

\end{izrek}
\vspace{0.5cm}

% *************************************************************************************************

\subsection{Izlimitirani integrali s parametrom}
\vspace{0.5cm}

\begin{definicija}

Integral $F(y) = \mathlarger{\int_{a}^{\infty} f(x, y) dx}$ je \textit{enakomerno konvergenten} za $y \in S \subseteq \mathbb{R}$, če za $\forall \varepsilon > 0 ~\exists M \in \mathbb{R}$, da za $\forall b \geq M$ in $\forall y \in S$ velja
$$\left| \int_{b}^{\infty} f(x, y) dx \right| < \varepsilon.$$ 

Za razliko od navadne konvergence mora tukaj obstajati tak $M$, ki je istočasno ustrezen za $\forall y \in S$, torej je $M = M_{\varepsilon}$ odvisen le od $\varepsilon$, ne pa tudi od $y$. Pri navadni konvergenci bi bil veljalo $M = M_{\varepsilon,y}$.

\end{definicija}
\vspace{0.5cm}

\begin{trditev}

Če je $f$ \textit{zvezna} funkcija na pasu $P = [a, \infty) \times [c, d]$ in integral
$$F(y) = \int_{a}^{\infty} f(x, y) dx$$
\textit{enakomerno konvergenten} za $y \in [c, d]$, je $F$ \textit{zvezna funkcija} na $[c, d]$.

\end{trditev}
\vspace{0.5cm}

% *************************************************************************************************

\subsection{Dvojni in dvakratni integrali}
\vspace{0.5cm}

\begin{definicija}

Naj bo $P = [a, b] \times [c, d]$ in $f: P \rightarrow \mathbb{R}$ funkcija. Delitev $D_{[a, b]}$ intervala $[a, b]$ je določena z zaporedjem točk
$$a = x_0 < x_1 < \ldots < x_m = b.$$
Delitev $D_{[a, b]}$ skupaj s poljubno delitvijo $D_{[c, d]}$ intervala $[c, d]$, določeno z 
$$c = y_0 < y_1 < \ldots < y_n = d,$$
določa neko delitev pravokotnika $P$ na manjpe pravokotnike
$$P_{i,j} = [x_{i-1}, x_i] \times [y_{i-1}, y_i], ~(i = 1, \ldots, m; ~j = 1, \ldots, n).$$

Naj bo 
$$m_{i,j} = \inf_{(x,y) \in P_{i,j}} f(x, y),$$
$$M_{i,j} = \sup_{(x,y) \in P_{i,j}} f(x, y).$$
Z $\Delta_{i,j} p = \Delta_i x \cdot \Delta_j y = (x_i - x_{i-1})(y_j - y_{j-1})$ označimo ploščino pravokotnika $P_{i,j}$. 
Vsoto
$$\underline{S}_D = \sum_{i=1}^{m} \sum_{j=1}^{n} m_{i,j} \Delta_{i,j} p$$
imenujemo \textit{spodnja}, vsoto
$$\overline{S}_D = \sum_{i=1}^{m} \sum_{j=1}^{n} M_{i,j} \Delta_{i,j} p$$
pa \textit{zgornja Riemannova vsota} funkcije $f$ pri delitvi $D$.

\end{definicija}
\vspace{0.5cm}

\begin{lema}

Če je $N$ nadaljevanje delitve $D$ pravokotnika $P$, za spodnje in zgornje Riemannove vsote poljubne omejene funkcije $f: P \rightarrow \mathbb{R}$ velja
$$\underline{S}_N \geq \underline{S}_D ~\text{in}~ \overline{S}_N \leq \overline{S}_D.$$

\end{lema}
\vspace{0.5cm}

\begin{definicija}

Omejena funkcija $f: P \rightarrow \mathbb{R}$ je na pravokotniku $P$ \textit{integrabilna v Riemannovem smislu}, če velja
$$\underline{S} = \overline{S},$$
kjer je $\underline{S}$ supermum njenih \textit{spodnjih}, $\overline{S}$ pa infimum njenih \textit{zgornjih} Riemannovih vsot. Tedaj skupno vrednost $\underline{S} = \overline{S}$ označimo kot 
$$\iint_P f(x, y) dp,$$
kjer pomeni $dp = dxdy$ \textit{ploščinski element}, in jo imenujemo \textit{dvojni integral funkcije $f$ po pravokotniku $P$}.

\end{definicija}
\vspace{0.5cm}

\begin{izrek}

\textit{Zvezna funkcija} $f$ na pravokotniku $P = [a, b] \times [c, d]$ je \textit{integrabilna} in velja
\begin{equation}
\int_{a}^{b} \left( \int_{c}^{d} f(x, y) dy \right) dx = \iint_P f(x, y) dp = \int_{c}^{d} \left( \int_{a}^{b} f(x, y) dx \right) dy.
\end{equation}

Enak zaključek velja tudi za funkcijo $f$, ki ni nujno zvezna, če je $N$ množica njenih točk nezveznosti taka, da jo za $\forall \varepsilon > 0$ lahko pokrijemo s kakim zaporedjem pravokotnikov, katerih vsota ploščin je pod $\varepsilon$. Tedaj pravimo, da ima $N$ mero $0$.

\end{izrek}
\vspace{0.5cm}

\begin{posledica}

Za funkcijo $f$, ki je na pravokotniku $P$ \textit{integrabilna} v Riemannovem smislu, konvergirajo Riemannove vsote $S$ proti $\mathlarger{\iint_P f(x, y) dp}$, ko gredo velikosti delilnih pravokotnikov (njihove diagonale) proti $0$. \\

Natančneje: za $\forall \varepsilon > 0 ~\exists \delta > 0$, da je 
$$\left| S - \iint_P f(x, y) dp \right| < \varepsilon$$
za vsako Riemannovo vsoto funkcije $f$ pri vsaki delitvi pravokotnika $P$, kjer si dolžine diagonal pod $\delta$.

\end{posledica}
\vspace{0.5cm}

% *************************************************************************************************

\subsection{Integriranje in odvajanje integralov s parametrom}
\vspace{0.5cm}

\begin{izrek}

Naj bo $f$ \textit{zvezna} na pasu $[a, \infty) \times [c, d]$. Če je integral $\mathlarger{\int_a^{\infty} f(x, y) dx}$ \textit{enakomerno konvergenten} za $y \in [c, d]$, potem je
$$\int_c^d \int_a^{\infty} f(x, y) dx ~dy = \int_a^{\infty} \int_c^d f(x, y) dy ~dx.$$

\end{izrek}
\vspace{0.5cm}

\begin{izrek}

Naj bosta $f$ in $\cfrac{\partial f}{\partial y}$ \textit{zvezni} na pasu $[a, \infty) \times [c, d]$, naj bo integral
$$F(y) = \int_a^{\infty} f(x, y) dx$$
\textit{konvergenten} za $y \in [c, d]$ in naj bo integral
$$\int_a^{\infty} \frac{\partial f}{\partial y}(x, y) dx$$
\textit{enakomerno konvergenten} na $[c, d]$. Potem je $F$ \textit{odvedljiva} funkcija in velja
$$F'(y) = \frac{d}{dy} \int_a^{\infty} f(x, y) dx = \int_a^{\infty} \frac{\partial f}{\partial y}(x, y) dx.$$

\end{izrek}
\vspace{0.5cm}

\begin{izrek}[Kriterij za ugotavljanje enakomerne konvergence]
~ \\
Integral $\mathlarger{\int_a^{\infty} f(x, y) dx} = F(y)$ je \textit{enakomerno konvergenten} na $S$ natanko tedaj, ko za $\forall \varepsilon > 0 ~\exists N \in \mathbb{R}$, da za poljubna $d > b \geq N$ in za $\forall y \in S$ velja
$$\left| \int_b^d f(x, y) dx \right| < \varepsilon.$$

\end{izrek}
\vspace{0.5cm}

\begin{posledica}

Če je $|f(x, y)| \leq g(x, y)$ za $\forall (x, y) \in [a, \infty) \times [c, d]$ in je integral $\mathlarger{\int_a^b g(x, y) dx}$ \textit{enakomerno konvergenten} na $[c, d]$, je \textit{enakomerno konvergenten} tudi integral $\mathlarger{\int_a^b f(x, y) dx}$.

\end{posledica}
\vspace{0.5cm}

\begin{izrek}[2. izrek o povprečju]

Naj bo $f$ \textit{integrabilna}, $g$ pa \textit{nenegativna padajoča (odvedljiva)} funkcija na intervalu $[a, b]$. Potem $\exists \xi \in [a, b]$, da je
$$\int_a^b f(x) g(x) dx = g(a) \int_a^{\xi} f(x) dx.$$

\end{izrek}
\vspace{0.5cm}

% *************************************************************************************************

\subsection{Eulerjevi funkciji $\Gamma$ in $B$}
\vspace{0.5cm}

\begin{definicija}[Funkcija $\Gamma$]

Na poltraku $x > 0$ je funkcija $\Gamma$ definirana z
\begin{equation}
	\Gamma(x) = \int_0^{\infty} t^{t-1} e^{-t} dt.
\end{equation}

\end{definicija}
\vspace{0.5cm}

\begin{trditev}[Rekurzivna formula]

Za $\forall x > 0$ velja
$$\Gamma(x + 1) = x \Gamma(x).$$

\end{trditev}
\vspace{0.5cm}

\begin{posledica}

$\Gamma(n + 1) = n!$ za $\forall n \in \mathbb{N}$ \\

\noindent To nam namiguje, naj definiramo
$$x! := \Gamma(x + 1) ~\text{za}~ \forall n \in \mathbb{N}.$$
Rekurzivna formula nam omogoča, da razširimo definicijsko območje funkcije $\Gamma$. Če je namreč $x \in (-1, 0)$, je $x + 1 \in  (0, 1)$, zato je vrednost $\Gamma(x + 1)$ že definiramo in lahko postavimo
$$\Gamma := \frac{\Gamma(x+1)}{x}.$$
S ponavljanjem rekurzivne formule dobimo
\begin{equation} \label{eq:2}
	\Gamma(x) = \frac{\Gamma(x+n)}{x(x+1)\ldots(x+n-1)}.
\end{equation}
Za $\forall x \in \mathbb{R}$, ki ni negativno celo število ali $0$, lahko izberemo tak najmanjši $n \in \mathbb{N}$, da je $(x + n) > 0$; tedaj je vrednost $\Gamma(x + n)$ že definirana in lahko $\Gamma(x)$ definiramo s formulo (\ref{eq:2}).

\end{posledica}
\vspace{0.5cm}

\begin{definicija}

Funkcija beta je definirana kot
\begin{equation} \label{eq:3}
	B(x, y) = \int_0^1 t^{x-1} (1-t)^{y-1} dt, ~~(x>0, y>0).
\end{equation}
Lahko se je prepričati, da je integral v (\ref{eq:3}) konvergenten, če je $x > 0$ in $y > 0$. \\

Z vpeljavo nove integracijske spremenljivke $t = \sin^2{\varphi}$ lahko definicijo funkcije $B$ zapišemo tudi kot
\begin{equation} \label{eq:4}
	B(x, y) = 2 \int_0^{\frac{\pi}{2}} \sin^{2x-1}{\varphi} \cos^{2y-1}{\varphi} d\varphi.
\end{equation}

\end{definicija}
\vspace{0.5cm}

\begin{trditev}

Za poljubna pozitivna $x, y$ je 
\begin{equation} \label{eq:5}
	B(x, y) = \frac{\Gamma(x)\Gamma(y)}{\Gamma(x+y)}
\end{equation}

\end{trditev}
\vspace{0.5cm} 

\begin{izrek}[Stirlingova formula]

$$\lim_{n \rightarrow \infty} \frac{n!}{\sqrt{n} (\frac{n}{e})^n} = \sqrt{2 \pi}$$

\end{izrek}
\vspace{0.5cm}

\begin{trditev}[Wallisova formula]

$$\lim_{n \rightarrow \infty} \frac{1}{2n+1} \prod_{j=1}^n \left( \frac{2j}{2j-1} \right)^2 = \frac{\pi}{2}$$

\end{trditev}
\vspace{0.5cm}

% *************************************************************************************************

\pagebreak

% #################################################################################################

\section{VEČKRATNI INTEGRALI}
\vspace{0.5cm}

\begin{definicija}

Naj bo $f: K \rightarrow \mathbb{R}$ \textit{omejena} funkcija, definirana na kvadru $K = [a, b] \times [c, d] \times [e, g]$ v $\mathbb{R}^3$. Vse tri intervale $[a, b]$, $[c, d]$ in $[e, g]$ razdelimo na podintervale z delilnimi točkami:
$$a = x_0 < \ldots < x_{i-1} < x_i < \ldots < x_m = b,$$
$$c = y_0 < \ldots < y_{j-1} < y_j < \ldots < y_n = d,$$
$$e = z_0 < \ldots < z_{k-1} < z_k < \ldots < z_p = g.$$ 
S tem razdelimo kvader $K$ na manjše podkvadre 
$$K_{i,j,k} = [x_{i-1}, x_i] \times [y_{j-1}, y_j] \times [z_{k-1}, z_k];$$
to delitev imenujemo $D$. Označimo
$$m_{i,j,k} = \inf_{(x,y,z) \in K_{i,j,k}} f(x, y, z),$$
$$M_{i,j,k} = \sup_{(x,y,z) \in K_{i,j,k}} f(x, y, z)$$
ter tvorimo \textit{spodnjo} in \textit{zgornjo} Riemannovo vsoto pri tej delitvi:
$$\underline{S}_D = \sum_{i=1}^m \sum_{j=1}^n \sum_{k=1}^p m_{i,j,k} \Delta_{i,j,k} V,$$
$$\overline{S}_D = \sum_{i=1}^m \sum_{j=1}^n \sum_{k=1}^p M_{i,j,k} \Delta_{i,j,k} V,$$
kjer je 
$$\Delta_{i,j,k} V = \Delta_i x \Delta_j y \Delta_k z = (x_i - x_{i-1})(y_j - y_{j-1})(z_k - z_{k-1})$$
prostornina kvadra $K_{i,j,k}$. Končno naj bo
$$\underline{S} = \sup_D \underline{S}_D ~~\text{in}~~ \overline{S} = \inf_D \overline{S}_D,$$
kjer teče $D$ po vseh takih delitvah kvadra $K$. Če je $\underline{S} = \overline{S}$ pravimo, da je funkcija $f$ \textit{integrabilna} na kvadru $K$ in skupno vrednost $\underline{S} = \overline{S}$ označimo kot
$$\iiint_K f(x, y, z) ~dV$$
ter jo imenujemo \textit{trojni} (Riemannov) integral funkcije $f$. \\

\end{definicija}
\vspace{0.5cm}

\begin{definicija}

Naj bo $\Omega$ poljubna omejena podmnožica v $\mathbb{R}^n$ ($n = 1, 2, 3, \ldots$), $f: \Omega \rightarrow \mathbb{R}$ pa omejena funkcija. Izberimo kvader $K$ oblike $K = [a, b] \times [c, d] \times \ldots$, ki naj vsebuje $\Omega$, definirajmo funkcijo $f_K: K \rightarrow \mathbb{R}$ kot
$$f(x) = 
\begin{cases}
	f(x, y, \ldots) &; ~(x, y, \ldots) \in \Omega \\
	0 &; ~(x, y, \ldots) \in K \setminus \Omega
\end{cases}$$ 
ter večkratni integral $\mathlarger{\idotsint_\Omega f(x, y, \ldots) ~dV}$ kot
$$\idotsint_\Omega f(x, y, \ldots) ~dV = \idotsint_K f_K(x, y, \ldots) ~dV.$$

\end{definicija}
\vspace{0.5cm}

\begin{trditev}

Če ima presek \textit{omejenih} množic $\Omega_1$ in $\Omega_2$ v $\mathbb{R}^2$ (ali v $\mathbb{R}^n$) mero 0, potem je
$$\idotsint_{\Omega_1 \cup \Omega_2} f(x_1, \ldots, x_n) ~dV = \idotsint_{\Omega_1} f(x_1, \ldots, x_n) ~dV + \idotsint_{\Omega_2} f(x_1, \ldots, x_n) ~dV$$

\end{trditev}
\vspace{0.5cm}

\begin{trditev}

Naj bosta $f_1$ in $f_2$ \textit{zvezni} funkciji na $\Omega$ ter $c_1$ in $c_2$ poljubni konstanti. Potem velja
$$\idotsint_{\Omega} (c_1 f_1 + c_2 f_2) ~dV = c_1 \idotsint_{\Omega} f_1 ~dV + c_2 \idotsint_{\Omega} f_2 ~dV.$$

\end{trditev}
\vspace{0.5cm}

\begin{trditev}

Če je $f \leq g$, potem je $\mathlarger{\iint_\Omega f ~dp \leq \iint_\Omega g ~dp}$ in podobno za večkratne integrale. Če je torej funkcija $f$ \textit{omejena} na mnoćici $\Omega$ navzgor s konstanto $M$, navzdol pa s konstanto $m$, potem velja
$$m p_\Omega \leq \iint_\Omega f dp \leq M p_\Omega,$$
kjer je $p_\Omega$ ploščina množice $\Omega$.

\end{trditev}
\vspace{0.5cm}

\begin{trditev}

$$\left| \iint_\Omega f ~dp \right| \leq \iint_\Omega |f| ~dp$$

\end{trditev}
\vspace{0.5cm}

\begin{trditev}

Naj bo $\Omega$ \textit{kompaktna} množica v $\mathbb{R}^2$, katere rob sestoji iz končno mnogo krivulj oblike $\vec:[a, b] \rightarrow \mathbb{R}^2$ za kake \textit{zvezno odvedljive} funkcije $\vec{r}$ in kake intervale $[a, b]$. Izberimo pravokotnik $P$, ki vsebuje $\Omega$, in naj bo $D$ poljubna delitev tega pravokotnika s premicami, vzporednimi koordinatnima osema. V vsakem od tistih delilnih pravokotnikov $P_k$ delitve $D$, ki sekajo $\Omega$, izberemo točko $\vec{r}_k \in P_k \cap \Omega$, označimo z $\Delta_k p$ ploščino pravokotnika $P_k$ in tvorimo Riemannovo vsoto
$$S_D(f) = \sum_k f(\vec{r}_k) \Delta_k p,$$
kjer teče indeks le po tistih delilnih pravokotnikih $P_k$, ki sekajo $\Omega$. Za vsako zvezno funkcijo $f$ na $\Omega$ je integral $\mathlarger{\iint_\Omega f(\vec{r}) ~dp}$ enak limiti vsot $S_D(f)$, ko gredo velikosti vseh delilnih pravokotnikov (torej največja diagonala vseh delilnih pravokotnikov) proti $0$. Natančneje, za $\forall \varepsilon > 0 ~\exists \delta > 0$, da je
$$\left| S_D(f) - \iint_\Omega f(\vec{r}) ~dp \right| < \varepsilon,$$
če je maksimalna diagonalna delilnih pravokotnikov $P_k$ manjša od $\delta$.

\end{trditev}
\vspace{0.5cm}

\begin{posledica}

Naj bo $\Omega$ podana kot
\begin{equation} \label{eq:6}
	\Omega = \{ (x, y) \in \mathbb{R}^2 \mid g_1(x) \leq y \leq g_2(x), ~a \leq x \leq b \},
\end{equation}
kjer sta $g_1$ in $g_2$ \textit{zvezni} funkciji na intervalu $[a, b]$ in $g_1(x) \leq g_2(x)$ za $\forall x \in [a, b]$. Naj bosta $M_1$ in $M_2$ taki števili, da pravokotnik $P = [a, b] \times [M_1, M_2]$ vsebuje množico $\Delta$ (torej $M_1 \leq g_1(x) \leq g_2(x) \leq M_2$ za $\forall x \in [a, b]$). Po definiciji imamo potem za vsako \textit{zvezno} funkcijo $f: \Omega \rightarrow \mathbb{R}$:
$$\iint_\Omega f(x, y) ~dp = \iint_P f_P(x, y) ~dp,$$
kjer je $f_P$ funkcija na $P$, definirana z 
$$f_P(x, y) = 
\begin{cases}
	f(x, y) &; ~(x, y) \in \Omega \\
	0 &; ~(x, y) \in P \setminus \Omega.
\end{cases}$$
Po zgornjem izreku pa je
$$\iint_P f_P(x, y) ~dp = \int_a^b \left( \int_{M_1}^{M_2} f(x, y) ~dy \right) dx = \int_a^b \left( \int_{g_1(x)}^{g_2(x)} f(x, y) ~dy \right) dx,$$
kjer smo upoštevali, da je funkcija $f_P$ enaka $0$ izven $\Omega$ in zato $\mathlarger{\int_{M_1}^{M_2} f_P(x, y) ~dy = \int_{g_1(x)}^{g_2(x)} ~dy}$. Torej velja naslednja trditev:

\end{posledica}
\vspace{0.5cm}

\begin{trditev}

Za vsako \textit{zvezno} funkcijo $f$ na množici $\Omega$, definirani kot (\ref{eq:6}), velja
$$\iint_\Omega f(x, y) ~dp = \int_a^b \left( \int_{g_1(x)}^{g_2(x)} f(x, y) ~dy \right) dx.$$

\end{trditev}
\vspace{0.5cm}

\begin{trditev}

Za območja $\Omega$, podana kot \\$\Omega = \{ (x, y, z) \in \mathbb{R}^3 \mid g_1(x, y) \leq z \leq g_2(x, y), ~(x, y) \in \Lambda \}$, in (skoraj povsod) \textit{zvezne} funkcije $f$ na njih velja
$$\iiint_\Omega f(x, y, z) ~dV = \iint_\Lambda \left( \int_{g_1(x)}^{g_2(x)} f(x, y, z) ~dz \right) dp.$$

\end{trditev}
\vspace{0.5cm}

% *************************************************************************************************

\subsection{Cilindrične ali valjne koordinate}
\vspace{0.5cm}

\begin{definicija}

Lega točke $(x, y, z)$ v prostoru $\mathbb{R}$ je določena s koordinato $z$ in polarnima koordinatama $r, \varphi$ njene projekcije $(x, y, 0)$ na ravnino $x,y$. Trojko $\varphi, r, z$ imenujemo \textit{cilindrične} ali \textit{valjne} koordinatne točke. S kartezičnimi koordinatami so povezane prek enakosti
$$x = r \cos{\varphi}, ~~y = r \sin{\varphi}, ~~z = z.$$
Pri tem lahko $r$ zavzame vse nenegativne vrednosti, $z$ vse realne vrednosti, $\varphi$ pa na intervalu $[0, 2\pi)$. Za dano točko $T$ pomeni $r$ njeno razdaljo od osi $z$, ki je enaka razdalji projekcije točke $T$ na ravnino $x,y$ od koordinatnega izhodišča.

\end{definicija}
\vspace{0.5cm}

\begin{posledica}[Koordinatne ploskve]
~\\
\begin{itemize}
	\item Ploskve $z = \text{konstanta}$ so ravnine, vzporedne z ravnino $x,y$
	\item Ploskve $r = \text{konstanta}$ so neskončni valji, katerih os je os $z$
	\item Ploskve $\varphi = \text{konstanta}$ pa so polravnine
\end{itemize}

\end{posledica}
\vspace{0.5cm}

% *************************************************************************************************

\subsection{Sferične koordinate}
\vspace{0.5cm}

\begin{definicija}

\textit{Sferične} ali \textit{krogelne koordinate} točke $T(x, y, z)$ so:
\begin{itemize}
	\item $R = \sqrt{x^2 + y^2 + z^2}$ razdalja od izhodišča
	\item $\theta$ kot, ki ga vektor $\vec{0T}$ oklepa s pozitivnim poltrakom osi $z$
	\item $\varphi$ kot, ki ga pravokotna projekcija vektorja $\vec{0T}$ na ravnino $x,y$ oklepa s pozitivnim poltrakom osi $x$
\end{itemize} 
Naj bo kot doslej r, razdalja $T$ od osi $z$. Potem je $r = R \sin{\theta}$ in
$$x = R \sin{\theta} \cos{\varphi}, ~~y = R \sin{\theta} \sin{\varphi}, ~~z = R \cos{\theta}.$$
Tukaj lahko zavzame kot $\theta$ vrednosti na intervalu $[0, \pi]$ ($0$ je na pozitivnem, $\pi$ pa na negativnem poltraku osi $z$), kot $\varphi$ pa vrednosti na intervalu $[0, 2\pi)$. \\

\noindent Volumni element v sferičnih koordinatah je
$$dV = R^2 \sin{\theta} ~dR ~d\theta ~d\varphi.$$
Od tod sledi, da lahko trojni integral po telesu $\Omega$, ki je opisano kot
$$\Omega = \{ (x, y, z) \in \mathbb{R}^3: ~g_1(\varphi, \theta) \leq R \leq g_2(\varphi, \theta), ~(\varphi, \theta) \in \Lambda \},$$
kjer sta $g_1 \leq g_2$ \textit{zvezni} funkciji na množici $\Lambda \subset \mathbb{R}^2$, izrazimo kot
$$\iiint_{\Omega} f(x, y, z) ~dV = \iint_{\Lambda} \left( \int_{g_1(\varphi, \theta)}^{g_2(\varphi, \theta)} f(R \sin{\theta} \cos{\varphi}, ~R \sin{\theta} \sin{\varphi}, ~R \cos{\theta}) R^2 \sin{\theta} ~dR \right) d\theta ~d\varphi.$$

\end{definicija}
\vspace{0.5cm}

\begin{posledica}[Koordinatne ploskve]
~\\
\begin{itemize}
	\item Ploskve $R = \text{konstanta}$ so sfere
	\item Ploskve $\theta = \text{konstanta}$ so stožci
	\item Plosvke $\varphi = \text{konstanta}$ so polravnine
\end{itemize}

\end{posledica}
\vspace{0.5cm}


% *************************************************************************************************

\subsection{Splošne koordinate}
\vspace{0.5cm}

\begin{definicija}

Naj bo $V$ \textit{odprta} podmožica v ravnini. Vlogo splošnih koordinat na $V$ lahko igra vsak tak par funkcij
$$u = u(x, y), ~~v = v(x, y)$$
na $V$, da iz $(x, y) \neq (x_1, y_2)$ sledi  $(u(x, y), v(x, y)) \neq (u(x_1, y_1), v(x_1, y_1))$, kar pomeni, da je točka $(x, y)$ enolično določena s parom $(u(x, y), v(x, y))$. Drugače povedano, vektorska funkcija 
$$F: V \rightarrow \mathbb{R}^2, ~F(x, y) = (u(x, y), v(x, y))$$
mora biti \textit{injektivna}. Zavoljo diferencialnega računa  predpostavimo, da sta funkciji $u$ in $v$ \textit{zvezno odvedljivi}. Iz \textit{izreka o inverzni preslikavi} vemo, da potem obrnljivost Jacobijeve matrike $F'(x, y)$ preslikave $F$ zagotavlja \textit{injektivnost} preslikave $F$ v okolici točke $(x, y)$, ne pa na celem definicijskem območju $V$, zato jo je treba posebej privzeti. Tedaj je pri pogoju, da je $F'(x, y)$ \textit{obrnljiva} matrika za $\forall (x, y) \in V$, preslikava $F$ dejansko \textit{bijekcija} na odprto množico $U := F(V)$, inverzna preslikava
$$G := F^{-1}: U \rightarrow V$$
pa je tudi \textit{zvezno odvedljiva} in 
$$G'(\vec{q}) = (F'(G(\vec{q})))^{-1} ~\text{za}~ \forall \vec{q} \in U.$$
Pri fiksnih $u_0$ in $v_0$ imenujemo krivulje $u = (x, y)$ in $v = (x, y)$ \textit{koordinatne} krivulje.

\end{definicija}
\vspace{0.5cm}

\begin{izrek}

Naj bo $G: U \rightarrow V$ taka \textit{zvezno odvedljiva bijekcija}, kjer sta $U$ in $V$ \textit{odprti} podmnožici v $\mathbb{R}$, da je $\det{G'(\vec{r})} \neq 0$ za $\forall \vec{r} \in U$. Označimo $\vec{r} = (u, v)$ in $G(u, v) = (x(u, v), y(u, v))$. Naj bo $\Omega$ \textit{kompaktna} podmnožica v $V$, katere rob naj sestoji iz končno mnogo \textit{zvezno odvedljivih} krivulj (oz. naj ima mero $0$), $f$ pa naj bo \textit{zvezna} funkcija na $V$ (razen morda na množici z mero $0$). Potem je
$$\iint_{\Omega} f(x, y) ~dx ~dy ~=~ \iint_{G^{-1}(\Omega)} f(x(u, v), y(u, v)) \left| \frac{\partial (x, y)}{\partial (u, v)}(u, v) \right| du ~dv,$$
kjer je 
$$\frac{\partial(x, y)}{\partial (u, v)}(u, v) ~=~ \mathlarger{\det{G'(u, v)} ~=~ \det{\begin{bmatrix}
\frac{\partial x}{\partial u}(u, v) & \frac{\partial x}{\partial v}(u, v) \\
\frac{\partial y}{\partial u}(u, v) & \frac{\partial y}{\partial v}(u, v)
\end{bmatrix}}}.$$

\end{izrek}
\vspace{0.5cm}

\begin{lema}

Naj bo $L: \mathbb{R}^2 \rightarrow \mathbb{R}^2$ \textit{obrnljiva linearna} preslikava in $\Lambda$ paralelogram. Potem med ploščinama paralelograma $\Lambda$ in $L(\Lambda)$ velja zveza
\begin{equation} \label{eq:7}
	p_{L(\Lambda)} = |\det{L}|_{p_{\Lambda}}.
\end{equation}
Enaka povezava velja tudi za vsako \textit{kompaktno} podmnožico $\Lambda$ v $\mathbb{R}^2$ oziroma za vsako \textit{ravninsko} podmnožico, za katero je ploščina definirana.

\end{lema}
\vspace{0.5cm}

\begin{lema}

Naj bo $G: U \rightarrow \mathbb{R}^2$ \textit{zvezno odvedljiva injektivna} preslikava s \textit{povsod obrbljivim} odvodom $G'(u, v)$, definirana na \textit{odprti} množici $U$, $K$ \textit{kompaktna} podmnožica v $U$, $\Lambda = \{ (u, v) \in \mathbb{R}^2; ~|u - a| \leq h, ~|v - b| \leq h \}$ pa kvadrat s središčem $(a, b)$ in stranico dolžine $2h$, vsebovan v $K$. Označimo
$$L ~=~ \mathlarger{\det{G'(a, b)} ~=~ \det{\begin{bmatrix}
\frac{\partial x}{\partial u}(a, b) & \frac{\partial x}{\partial v}(a, b) \\
\frac{\partial y}{\partial u}(a, b) & \frac{\partial y}{\partial v}(a, b)
\end{bmatrix}}}$$
in naj bo $A$ preslikava, definirana z $A(u, v) = G(a, b) + L(u-a,v-b)$. Potem za $\forall \varepsilon > 0 ~~\exists \delta > 0$ (neodvisen od izbire kvadrata), da za ploščini likov $G(\Lambda)$ in $A(\Lambda)$, ko je $h < \delta$, velja
$$|p_{G(\Lambda)} - p_{A(\Lambda)}| < \varepsilon p_{\Lambda}.$$
Ker se preslikavi $A$ in $L$ razlikujeta le za translacijo, lahko v tej oceni nadomestimo $A$ z $L$.

\end{lema}
\vspace{0.5cm}

% *************************************************************************************************

\pagebreak

% #################################################################################################

\end{document}
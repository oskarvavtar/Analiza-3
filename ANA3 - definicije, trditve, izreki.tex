\documentclass[11pt]{article}
\usepackage[utf8]{inputenc}
\usepackage[slovene]{babel}

\usepackage{amsthm}
\usepackage{amsmath, amssymb, amsfonts}

\theoremstyle{definition}
\newtheorem{definicija}{Definicija}[section]

\theoremstyle{corollary}
\newtheorem{trditev}{Trditev}[section]

\newtheorem{lema}{Lema}
\newtheorem{izrek}{Izrek}
\newtheorem*{dokaz}{Dokaz}

\title{Analiza 3 - definicije, trditve in izreki}
\author{Oskar Vavtar}
\date{2020/21}

\begin{document}
\maketitle
\pagebreak
\tableofcontents
\pagebreak

\section{PARAMETRIČNO PODANE KRIVULJE}

\vspace{0.5cm}	

\begin{trditev}

Če je $\vec{r}$ \textit{odvedljiva} vektorska funkcija (njene komponente $x$, $y$ in $z$ so odvedljive funkcije spremenljivke $t$), potem je
$$\dot{\vec{r}}(t_0) = (\dot{x}(t_0), \dot{y}(t_0), \dot{z}(t_0))$$
\textit{tangentni vektor} na krivuljo $t \mapsto \vec{r}(t)$ v točki $\vec{r}(t_0)$, če velja $\dot{\vec{r}}(t_0) \neq 0$.
	
\end{trditev}

\vspace{0.5cm}

\begin{trditev}

Če je $\vec{r}$ \textit{zvezno odvedljiva} vektorska funkcija na intervalu $[a, b]$ (za	$a < b$), je potem \textit{dolžina} krivulje, ki jo določa, enaka
$$s = \int_{a}^{b} \| \dot{\vec{r}}(t) \| dt.$$
To velja tudi za funkcijo, ki so le \textit{odsekoma zvezne}. Opazimo tudi, da je zgornja dolžina neodvisna od parametrizacije krivulje.	

\end{trditev}

\vspace{0.5cm}

\begin{trditev}

Naj bo $\vec{r}$ \textit{zvezno odvedljiva} vektorska funkcija, definirana na intervalu $[a, b]$ (za $a < b$) in naj bo $\psi: [a, b] \rightarrow [\alpha, \beta]$ \textit{zvezno odvedljiva bijekcija}, tako da $t = \psi(\tau)$ preteče interval $[a, b]$, ko $\tau$ preteče interval $[\alpha, \beta]$ (za $\alpha < \beta)$. Potem je
$$\int_{a}^{b} \| \dot{\vec{r}}(t) \| dt = \int_{\alpha}^{\beta} \| \frac{d}{d \tau} \vec{r}(\psi(\tau)) \| d\tau. $$

\end{trditev}

% *************************************************************************************************

\pagebreak

% #################################################################################################

\section{PLOSKVE}
\vspace{0.5cm}

% *************************************************************************************************

\subsection{Ploskve v $\mathbb{R}^3$}
\vspace{0.5cm}

\begin{definicija}[Ploskev]

Podmnožica $P \subseteq \mathbb{R}^3$ je \textit{ploskev}, če za $\forall$ točko $\vec{r} \in P$ $\exists$ taka okolica $H \subseteq \mathbb{R}^3$, da je $P \cap H$ graf kake zvezno odvedljive funkcije $\phi: D \rightarrow \mathbb{R}$, definirane na kaki \textit{odprti} podmnožici $D \subseteq \mathbb{R}^2$. \\
To pomeni, da se na $P \cap H$ ena od koordinat $x, y, z$ da \textit{enolično} izraziti kot funkcija preostalih, torej da je $P \cap H$ ene od oblik:
$$P \cap H = \{(x, y, \phi(x, y)) \mid (x, y) \in D\},$$
$$P \cap H = \{(x, \phi(x, z), y) \mid (x, z) \in D\},$$
$$P \cap H = \{(\phi(y, z), y, z) \mid (y, z) \in D\}.$$

\end{definicija}
\vspace{0.5cm}

\begin{trditev}[Izrek o implicitni funkciji]

Naj bo $g: \mathbb{R}^3 \rightarrow \mathbb{R}$ \textit{zvezno odvedljiva} funkcija in privzemimo, da je množica $P = g^{-1}(0)$ \textit{neprazna}. Če je
$$\nabla g(\vec{r}) \neq 0$$
za $\forall \vec{r} \in P$ je $P$ \textit{ploskev}. \\

Enačba oblike $\vec{r} = \vec{r}(t)$ ($t \in [a, b] \subseteq \mathbb{R}, ~a < b$) predstavlja krivuljo v $\mathbb{R}^3$. Privzeli bomo, da je pri tem $\vec{r}$ \textit{zvezno odvedljiva} funkcija spremenljivke $t$. Taka krivulja leži na ploskvi $P = g^{-1}(0)$ natanko tedaj, ko je $g(\vec{r}(t)) = 0$ za $\forall t \in [a, b]$. Ko to enakost odvajamo po $t$, dobimo
$$\nabla g(\vec{r}(t)) \cdot \dot{\vec{r}}(t) = 0.$$
Ta enakost pomeni, da je vektor $\nabla g(\vec{r}(t))$ pravokoten na tangentni vektor $\dot{\vec{r}}(t)$ krivulje v točki $\vec{r}(t)$. \\

Če sedaj izberemo poljubno točko $\vec{r}_0$ na ploskvi $P$ in opazujemo vse krivulje na ploskvi $P$, ki gredo skozi točko $\vec{r}_0$ (vsaka taka krivulja $\vec{r} = \vec{r}(t)$ zadošča pogoju $\vec{r}(t_0) = \vec{r}_0$ za kak $t_0$), vidimo, da je vektor $\nabla g(\vec{r}_0)$ pravokoten na tangentni vektor $\dot{\vec{r}}(t_0)$ vsake take krivulje. \\

To pomeni, da mora biti vektor $\nabla g(\vec{r}_0)$ pravokoten na ploskev $P$. To velja za $\forall$ točko $\vec{r}_0 \in P$.
 

\end{trditev}
\vspace{0.5cm}

\begin{definicija}[Normalni vektor]

Vektor $\nabla g(\vec{r})$ imenujemo \textit{normalni vektor} na ploskev $P = g^{-1}(0)$ v točki $\vec{r} \in P$. Ravnino $T_{\vec{r}}P$ z normalnim vektorjem $\nabla g(\vec{r})$ skozi točko $\vec{r}$ na ploskvi $P$ pa imenujemo \textit{tangentna ravnina} na ploskev $P$ v točki $\vec{r}$. \\

Tangentna ravnina na $P$ skozi točko $\vec{r}$ je torej vzporedna vsem tangentnim vektorjem v točki $\vec{r}$ na krivulje skozi $\vec{r}$ na ploskvi $P$.

\end{definicija}
\vspace{0.5cm}

\end{document}
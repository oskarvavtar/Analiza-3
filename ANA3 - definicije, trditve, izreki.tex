\documentclass[11pt]{article}
\usepackage[utf8]{inputenc}
\usepackage[slovene]{babel}

\usepackage{amsthm}
\usepackage{amsmath, amssymb, amsfonts}
\usepackage{relsize}

\theoremstyle{definition}
\newtheorem{definicija}{Definicija}[section]

\theoremstyle{definition}
\newtheorem{trditev}{Trditev}[section]

\theoremstyle{definition}
\newtheorem{izrek}{Izrek}[section]

\renewcommand{\qedsymbol}{$\blacksquare$}

\newtheorem{lema}{Lema}
\newtheorem*{posledica}{Posledica}
\newtheorem*{opomba}{Opomba}

\title{Analiza 3 - definicije, trditve in izreki}
\author{Oskar Vavtar \\
po predavanjih profesorja Bojana Magajne}
\date{2020/21}

\begin{document}
\maketitle
\pagebreak
\tableofcontents
\pagebreak

\section{PARAMETRIČNO PODANE KRIVULJE}
\vspace{0.5cm}	

\begin{trditev}

Če je $\vec{r}$ \textit{odvedljiva} vektorska funkcija (njene komponente $x$, $y$ in $z$ so odvedljive funkcije spremenljivke $t$), potem je
$$\dot{\vec{r}}(t_0) ~=~ (\dot{x}(t_0), \dot{y}(t_0), \dot{z}(t_0))$$
\textit{tangentni vektor} na krivuljo $t \mapsto \vec{r}(t)$ v točki $\vec{r}(t_0)$, če velja $\dot{\vec{r}}(t_0) \neq 0$.
	
\end{trditev}
\vspace{0.5cm}

\begin{trditev}

Če je $\vec{r}$ \textit{zvezno odvedljiva} vektorska funkcija na intervalu $[a, b]$ (za	$a < b$), je potem \textit{dolžina} krivulje, ki jo določa, enaka
$$s ~=~ \int_{a}^{b} \| \dot{\vec{r}}(t) \| dt.$$
To velja tudi za funkcijo, ki so le \textit{odsekoma zvezne}. Opazimo tudi, da je zgornja dolžina neodvisna od parametrizacije krivulje.	

\end{trditev}
\vspace{0.5cm}

\begin{trditev}

Naj bo $\vec{r}$ \textit{zvezno odvedljiva} vektorska funkcija, definirana na intervalu $[a, b]$ (za $a < b$) in naj bo $\psi: [a, b] \rightarrow [\alpha, \beta]$ \textit{zvezno odvedljiva bijekcija}, tako da $t = \psi(\tau)$ preteče interval $[a, b]$, ko $\tau$ preteče interval $[\alpha, \beta]$ (za $\alpha < \beta)$. Potem je
$$\int_{a}^{b} \| \dot{\vec{r}}(t) \| dt ~=~ \int_{\alpha}^{\beta} \| \frac{d}{d \tau} \vec{r}(\psi(\tau)) \| d\tau. $$

\end{trditev}
\vspace{0.5cm}

% *************************************************************************************************

\pagebreak

% #################################################################################################

\section{PLOSKVE V $\mathbb{R}^3$}
\vspace{0.5cm}

% *************************************************************************************************

\begin{definicija}[Ploskev]

Podmnožica $P \subseteq \mathbb{R}^3$ je \textit{ploskev}, če za vsako točko $\vec{r} \in P$ obstaja taka okolica $H \subseteq \mathbb{R}^3$, da je $P \cap H$ graf kake zvezno odvedljive funkcije $\phi: D \rightarrow \mathbb{R}$, definirane na kaki \textit{odprti} podmnožici $D \subseteq \mathbb{R}^2$. \\
To pomeni, da se na $P \cap H$ ena od koordinat $x, y, z$ da \textit{enolično} izraziti kot funkcija preostalih, torej da je $P \cap H$ ene od oblik:
$$P \cap H ~=~ \{(x, y, \phi(x, y)) \mid (x, y) \in D\},$$
$$P \cap H ~=~ \{(x, \phi(x, z), y) \mid (x, z) \in D\},$$
$$P \cap H ~=~ \{(\phi(y, z), y, z) \mid (y, z) \in D\}.$$

\end{definicija}
\vspace{0.5cm}

\begin{trditev}[Izrek o implicitni funkciji]

Naj bo $g: \mathbb{R}^3 \rightarrow \mathbb{R}$ \textit{zvezno odvedljiva} funkcija in privzemimo, da je množica $P = g^{-1}(0)$ \textit{neprazna}. Če je
$$\nabla g(\vec{r}) ~\neq~ 0$$
za $\forall \vec{r} \in P$ je $P$ \textit{ploskev}. \\

Enačba oblike $\vec{r} = \vec{r}(t)$ ($t \in [a, b] \subseteq \mathbb{R}, ~a < b$) predstavlja krivuljo v $\mathbb{R}^3$. Privzeli bomo, da je pri tem $\vec{r}$ \textit{zvezno odvedljiva} funkcija spremenljivke $t$. Taka krivulja leži na ploskvi $P = g^{-1}(0)$ natanko tedaj, ko je $g(\vec{r}(t)) = 0$ za $\forall t \in [a, b]$. Ko to enakost odvajamo po $t$, dobimo
$$\nabla g(\vec{r}(t)) \cdot \dot{\vec{r}}(t) ~=~ 0.$$
Ta enakost pomeni, da je vektor $\nabla g(\vec{r}(t))$ pravokoten na tangentni vektor $\dot{\vec{r}}(t)$ krivulje v točki $\vec{r}(t)$. \\

Če sedaj izberemo poljubno točko $\vec{r}_0$ na ploskvi $P$ in opazujemo vse krivulje na ploskvi $P$, ki gredo skozi točko $\vec{r}_0$ (vsaka taka krivulja $\vec{r} = \vec{r}(t)$ zadošča pogoju $\vec{r}(t_0) = \vec{r}_0$ za kak $t_0$), vidimo, da je vektor $\nabla g(\vec{r}_0)$ pravokoten na tangentni vektor $\dot{\vec{r}}(t_0)$ vsake take krivulje. \\

To pomeni, da mora biti vektor $\nabla g(\vec{r}_0)$ pravokoten na ploskev $P$. To velja za vsako točko $\vec{r}_0 \in P$.
 

\end{trditev}
\vspace{0.5cm}

\begin{definicija}[Normalni vektor]

Vektor $\nabla g(\vec{r})$ imenujemo \textit{normalni vektor} na ploskev $P = g^{-1}(0)$ v točki $\vec{r} \in P$. Ravnino $T_{\vec{r}}P$ z normalnim vektorjem $\nabla g(\vec{r})$ skozi točko $\vec{r}$ na ploskvi $P$ pa imenujemo \textit{tangentna ravnina} na ploskev $P$ v točki $\vec{r}$. \\

Tangentna ravnina na $P$ skozi točko $\vec{r}$ je torej vzporedna vsem tangentnim vektorjem v točki $\vec{r}$ na krivulje skozi $\vec{r}$ na ploskvi $P$.

\end{definicija}
\vspace{0.5cm}

% *************************************************************************************************

\pagebreak

% #################################################################################################

\section{INTEGRALI S PARAMETROM}
\vspace{0.5cm}

% *************************************************************************************************

\begin{definicija}[Integral s parametrom]

Naj bo $f$ \textit{zvezna funkcija} dveh spremenljivk, definirana na pravokotniku $P = [a, b] \times [c, d]$ ($a < b$, $c < d$). Integral
\begin{equation}
\label{eq:1}
	F(y) = \int_{a}^{b} f(x, y) dx
\end{equation}

je funkcija spremenljivke $y$. Tak integral imenujemo \textit{integral s parametrom} $y$.

\end{definicija}
\vspace{0.5cm}

\begin{trditev}

Če je $f$ \textit{zvezna funkcija} na pravokotniku $P = [a, b] \times [c, d]$, je funkcija $F$ (definirana z (\ref{eq:1})) \textit{zvezna} na intervalu $P$.

\end{trditev}
\vspace{0.5cm}

\begin{izrek}

Naj bo $f$ \textit{zvezna} na pravokotniku $P = [a, b] \times [c, d]$ in privzemimo, da obstaja parcialni odvod $\cfrac{\partial f}{\partial y}$, ki naj bo \textit{zvezen} na $P$. Potem je funkcija $F$ (podana z (\ref{eq:1})) \textit{odvedljiva} in velja 
\begin{equation}
	F'(y) = \frac{d}{dy} \int_{a}^{b} f(x, y) dx = \int_{a}^{b} \frac{\partial f}{\partial y} (x, y) dx.
\end{equation}

\end{izrek}
\vspace{0.5cm}

% *************************************************************************************************

\subsection{Izlimitirani integrali s parametrom}
\vspace{0.5cm}

\begin{definicija}

Integral $F(y) = \mathlarger{\int_{a}^{\infty} f(x, y) dx}$ je \textit{enakomerno konvergenten} za $y \in S \subseteq \mathbb{R}$, če za $\forall \varepsilon > 0 ~\exists M \in \mathbb{R}$, da za $\forall b \geq M$ in $\forall y \in S$ velja
$$\left| \int_{b}^{\infty} f(x, y) dx \right| < \varepsilon.$$ 

Za razliko od navadne konvergence mora tukaj obstajati tak $M$, ki je istočasno ustrezen za $\forall y \in S$, torej je $M = M_{\varepsilon}$ odvisen le od $\varepsilon$, ne pa tudi od $y$. Pri navadni konvergenci bi bil veljalo $M = M_{\varepsilon,y}$.

\end{definicija}
\vspace{0.5cm}

\begin{trditev}

Če je $f$ \textit{zvezna} funkcija na pasu $P = [a, \infty) \times [c, d]$ in integral
$$F(y) = \int_{a}^{\infty} f(x, y) dx$$
\textit{enakomerno konvergenten} za $y \in [c, d]$, je $F$ \textit{zvezna funkcija} na $[c, d]$.

\end{trditev}
\vspace{0.5cm}

% *************************************************************************************************

\subsection{Dvojni in dvakratni integrali}
\vspace{0.5cm}

\begin{definicija}

Naj bo $P = [a, b] \times [c, d]$ in $f: P \rightarrow \mathbb{R}$ funkcija. Delitev $D_{[a, b]}$ intervala $[a, b]$ je določena z zaporedjem točk
$$a = x_0 < x_1 < \ldots < x_m = b.$$
Delitev $D_{[a, b]}$ skupaj s poljubno delitvijo $D_{[c, d]}$ intervala $[c, d]$, določeno z 
$$c = y_0 < y_1 < \ldots < y_n = d,$$
določa neko delitev pravokotnika $P$ na manjpe pravokotnike
$$P_{i,j} = [x_{i-1}, x_i] \times [y_{i-1}, y_i], ~(i = 1, \ldots, m; ~j = 1, \ldots, n).$$

Naj bo 
$$m_{i,j} = \inf_{(x,y) \in P_{i,j}} f(x, y),$$
$$M_{i,j} = \sup_{(x,y) \in P_{i,j}} f(x, y).$$
Z $\Delta_{i,j} p = \Delta_i x \cdot \Delta_j y = (x_i - x_{i-1})(y_j - y_{j-1})$ označimo ploščino pravokotnika $P_{i,j}$. 
Vsoto
$$\underline{S}_D = \sum_{i=1}^{m} \sum_{j=1}^{n} m_{i,j} \Delta_{i,j} p$$
imenujemo \textit{spodnja}, vsoto
$$\overline{S}_D = \sum_{i=1}^{m} \sum_{j=1}^{n} M_{i,j} \Delta_{i,j} p$$
pa \textit{zgornja Riemannova vsota} funkcije $f$ pri delitvi $D$.

\end{definicija}
\vspace{0.5cm}

\begin{lema}

Če je $N$ nadaljevanje delitve $D$ pravokotnika $P$, za spodnje in zgornje Riemannove vsote poljubne omejene funkcije $f: P \rightarrow \mathbb{R}$ velja
$$\underline{S}_N \geq \underline{S}_D ~\text{in}~ \overline{S}_N \leq \overline{S}_D.$$

\end{lema}
\vspace{0.5cm}

\begin{definicija}

Omejena funkcija $f: P \rightarrow \mathbb{R}$ je na pravokotniku $P$ \textit{integrabilna v Riemannovem smislu}, če velja
$$\underline{S} = \overline{S},$$
kjer je $\underline{S}$ supermum njenih \textit{spodnjih}, $\overline{S}$ pa infimum njenih \textit{zgornjih} Riemannovih vsot. Tedaj skupno vrednost $\underline{S} = \overline{S}$ označimo kot 
$$\iint_P f(x, y) dp,$$
kjer pomeni $dp = dxdy$ \textit{ploščinski element}, in jo imenujemo \textit{dvojni integral funkcije $f$ po pravokotniku $P$}.

\end{definicija}
\vspace{0.5cm}

\begin{izrek}

\textit{Zvezna funkcija} $f$ na pravokotniku $P = [a, b] \times [c, d]$ je \textit{integrabilna} in velja
\begin{equation}
\int_{a}^{b} \left( \int_{c}^{d} f(x, y) dy \right) dx = \iint_P f(x, y) dp = \int_{c}^{d} \left( \int_{a}^{b} f(x, y) dx \right) dy.
\end{equation}

Enak zaključek velja tudi za funkcijo $f$, ki ni nujno zvezna, če je $N$ množica njenih točk nezveznosti taka, da jo za $\forall \varepsilon > 0$ lahko pokrijemo s kakim zaporedjem pravokotnikov, katerih vsota ploščin je pod $\varepsilon$. Tedaj pravimo, da ima $N$ mero $0$.

\end{izrek}
\vspace{0.5cm}

\begin{posledica}

Za funkcijo $f$, ki je na pravokotniku $P$ \textit{integrabilna} v Riemannovem smislu, konvergirajo Riemannove vsote $S$ proti $\mathlarger{\iint_P f(x, y) dp}$, ko gredo velikosti delilnih pravokotnikov (njihove diagonale) proti $0$. \\

Natančneje: za $\forall \varepsilon > 0 ~\exists \delta > 0$, da je 
$$\left| S - \iint_P f(x, y) dp \right| < \varepsilon$$
za vsako Riemannovo vsoto funkcije $f$ pri vsaki delitvi pravokotnika $P$, kjer si dolžine diagonal pod $\delta$.

\end{posledica}
\vspace{0.5cm}

% *************************************************************************************************

\subsection{Integriranje in odvajanje integralov s parametrom}
\vspace{0.5cm}

\begin{izrek}

Naj bo $f$ \textit{zvezna} na pasu $[a, \infty) \times [c, d]$. Če je integral $\mathlarger{\int_a^{\infty} f(x, y) dx}$ \textit{enakomerno konvergenten} za $y \in [c, d]$, potem je
$$\int_c^d \int_a^{\infty} f(x, y) dx ~dy = \int_a^{\infty} \int_c^d f(x, y) dy ~dx.$$

\end{izrek}
\vspace{0.5cm}

\begin{izrek}

Naj bosta $f$ in $\cfrac{\partial f}{\partial y}$ \textit{zvezni} na pasu $[a, \infty) \times [c, d]$, naj bo integral
$$F(y) = \int_a^{\infty} f(x, y) dx$$
\textit{konvergenten} za $y \in [c, d]$ in naj bo integral
$$\int_a^{\infty} \frac{\partial f}{\partial y}(x, y) dx$$
\textit{enakomerno konvergenten} na $[c, d]$. Potem je $F$ \textit{odvedljiva} funkcija in velja
$$F'(y) = \frac{d}{dy} \int_a^{\infty} f(x, y) dx = \int_a^{\infty} \frac{\partial f}{\partial y}(x, y) dx.$$

\end{izrek}
\vspace{0.5cm}

\begin{izrek}[Kriterij za ugotavljanje enakomerne konvergence]
~ \\
Integral $\mathlarger{\int_a^{\infty} f(x, y) dx} = F(y)$ je \textit{enakomerno konvergenten} na $S$ natanko tedaj, ko za $\forall \varepsilon > 0 ~\exists N \in \mathbb{R}$, da za poljubna $d > b \geq N$ in za $\forall y \in S$ velja
$$\left| \int_b^d f(x, y) dx \right| < \varepsilon.$$

\end{izrek}
\vspace{0.5cm}

\begin{posledica}

Če je $|f(x, y)| \leq g(x, y)$ za $\forall (x, y) \in [a, \infty) \times [c, d]$ in je integral $\mathlarger{\int_a^b g(x, y) dx}$ \textit{enakomerno konvergenten} na $[c, d]$, je \textit{enakomerno konvergenten} tudi integral $\mathlarger{\int_a^b f(x, y) dx}$.

\end{posledica}
\vspace{0.5cm}

\begin{izrek}[2. izrek o povprečju]

Naj bo $f$ \textit{integrabilna}, $g$ pa \textit{nenegativna padajoča (odvedljiva)} funkcija na intervalu $[a, b]$. Potem $\exists \xi \in [a, b]$, da je
$$\int_a^b f(x) g(x) dx = g(a) \int_a^{\xi} f(x) dx.$$

\end{izrek}
\vspace{0.5cm}

% *************************************************************************************************

\subsection{Eulerjevi funkciji $\Gamma$ in $B$}
\vspace{0.5cm}

\begin{definicija}[Funkcija $\Gamma$]

Na poltraku $x > 0$ je funkcija $\Gamma$ definirana z
\begin{equation}
	\Gamma(x) = \int_0^{\infty} t^{t-1} e^{-t} dt.
\end{equation}

\end{definicija}
\vspace{0.5cm}

\begin{trditev}[Rekurzivna formula]

Za $\forall x > 0$ velja
$$\Gamma(x + 1) = x \Gamma(x).$$

\end{trditev}
\vspace{0.5cm}

\begin{posledica}

$\Gamma(n + 1) = n!$ za $\forall n \in \mathbb{N}$ \\

\noindent To nam namiguje, naj definiramo
$$x! := \Gamma(x + 1) ~\text{za}~ \forall n \in \mathbb{N}.$$
Rekurzivna formula nam omogoča, da razširimo definicijsko območje funkcije $\Gamma$. Če je namreč $x \in (-1, 0)$, je $x + 1 \in  (0, 1)$, zato je vrednost $\Gamma(x + 1)$ že definiramo in lahko postavimo
$$\Gamma := \frac{\Gamma(x+1)}{x}.$$
S ponavljanjem rekurzivne formule dobimo
\begin{equation} \label{eq:2}
	\Gamma(x) = \frac{\Gamma(x+n)}{x(x+1)\ldots(x+n-1)}.
\end{equation}
Za $\forall x \in \mathbb{R}$, ki ni negativno celo število ali $0$, lahko izberemo tak najmanjši $n \in \mathbb{N}$, da je $(x + n) > 0$; tedaj je vrednost $\Gamma(x + n)$ že definirana in lahko $\Gamma(x)$ definiramo s formulo (\ref{eq:2}).

\end{posledica}
\vspace{0.5cm}

\begin{definicija}

Funkcija beta je definirana kot
\begin{equation} \label{eq:3}
	B(x, y) = \int_0^1 t^{x-1} (1-t)^{y-1} dt, ~~(x>0, y>0).
\end{equation}
Lahko se je prepričati, da je integral v (\ref{eq:3}) konvergenten, če je $x > 0$ in $y > 0$. \\

Z vpeljavo nove integracijske spremenljivke $t = \sin^2{\varphi}$ lahko definicijo funkcije $B$ zapišemo tudi kot
\begin{equation} \label{eq:4}
	B(x, y) = 2 \int_0^{\frac{\pi}{2}} \sin^{2x-1}{\varphi} \cos^{2y-1}{\varphi} d\varphi.
\end{equation}

\end{definicija}
\vspace{0.5cm}

\begin{trditev}

Za poljubna pozitivna $x, y$ je 
\begin{equation} \label{eq:5}
	B(x, y) = \frac{\Gamma(x)\Gamma(y)}{\Gamma(x+y)}
\end{equation}

\end{trditev}
\vspace{0.5cm} 

\begin{izrek}[Stirlingova formula]

$$\lim_{n \rightarrow \infty} \frac{n!}{\sqrt{n} (\frac{n}{e})^n} = \sqrt{2 \pi}$$

\end{izrek}
\vspace{0.5cm}

\begin{trditev}[Wallisova formula]

$$\lim_{n \rightarrow \infty} \frac{1}{2n+1} \prod_{j=1}^n \left( \frac{2j}{2j-1} \right)^2 = \frac{\pi}{2}$$

\end{trditev}
\vspace{0.5cm}

% *************************************************************************************************

\pagebreak

% #################################################################################################

\section{VEČKRATNI INTEGRALI}
\vspace{0.5cm}

\begin{definicija}

Naj bo $f: K \rightarrow \mathbb{R}$ \textit{omejena} funkcija, definirana na kvadru $K = [a, b] \times [c, d] \times [e, g]$ v $\mathbb{R}^3$. Vse tri intervale $[a, b]$, $[c, d]$ in $[e, g]$ razdelimo na podintervale z delilnimi točkami:
$$a = x_0 < \ldots < x_{i-1} < x_i < \ldots < x_m = b,$$
$$c = y_0 < \ldots < y_{j-1} < y_j < \ldots < y_n = d,$$
$$e = z_0 < \ldots < z_{k-1} < z_k < \ldots < z_p = g.$$ 
S tem razdelimo kvader $K$ na manjše podkvadre 
$$K_{i,j,k} = [x_{i-1}, x_i] \times [y_{j-1}, y_j] \times [z_{k-1}, z_k];$$
to delitev imenujemo $D$. Označimo
$$m_{i,j,k} = \inf_{(x,y,z) \in K_{i,j,k}} f(x, y, z),$$
$$M_{i,j,k} = \sup_{(x,y,z) \in K_{i,j,k}} f(x, y, z)$$
ter tvorimo \textit{spodnjo} in \textit{zgornjo} Riemannovo vsoto pri tej delitvi:
$$\underline{S}_D = \sum_{i=1}^m \sum_{j=1}^n \sum_{k=1}^p m_{i,j,k} \Delta_{i,j,k} V,$$
$$\overline{S}_D = \sum_{i=1}^m \sum_{j=1}^n \sum_{k=1}^p M_{i,j,k} \Delta_{i,j,k} V,$$
kjer je 
$$\Delta_{i,j,k} V = \Delta_i x \Delta_j y \Delta_k z = (x_i - x_{i-1})(y_j - y_{j-1})(z_k - z_{k-1})$$
prostornina kvadra $K_{i,j,k}$. Končno naj bo
$$\underline{S} = \sup_D \underline{S}_D ~~\text{in}~~ \overline{S} = \inf_D \overline{S}_D,$$
kjer teče $D$ po vseh takih delitvah kvadra $K$. Če je $\underline{S} = \overline{S}$ pravimo, da je funkcija $f$ \textit{integrabilna} na kvadru $K$ in skupno vrednost $\underline{S} = \overline{S}$ označimo kot
$$\iiint_K f(x, y, z) ~dV$$
ter jo imenujemo \textit{trojni} (Riemannov) integral funkcije $f$. \\

\end{definicija}
\vspace{0.5cm}

\begin{definicija}

Naj bo $\Omega$ poljubna omejena podmnožica v $\mathbb{R}^n$ ($n = 1, 2, 3, \ldots$), $f: \Omega \rightarrow \mathbb{R}$ pa omejena funkcija. Izberimo kvader $K$ oblike $K = [a, b] \times [c, d] \times \ldots$, ki naj vsebuje $\Omega$, definirajmo funkcijo $f_K: K \rightarrow \mathbb{R}$ kot
$$f(x) = 
\begin{cases}
	f(x, y, \ldots) &; ~(x, y, \ldots) \in \Omega \\
	0 &; ~(x, y, \ldots) \in K \setminus \Omega
\end{cases}$$ 
ter večkratni integral $\mathlarger{\idotsint_\Omega f(x, y, \ldots) ~dV}$ kot
$$\idotsint_\Omega f(x, y, \ldots) ~dV = \idotsint_K f_K(x, y, \ldots) ~dV.$$

\end{definicija}
\vspace{0.5cm}

\begin{trditev}

Če ima presek \textit{omejenih} množic $\Omega_1$ in $\Omega_2$ v $\mathbb{R}^2$ (ali v $\mathbb{R}^n$) mero 0, potem je
$$\idotsint_{\Omega_1 \cup \Omega_2} f(x_1, \ldots, x_n) ~dV = \idotsint_{\Omega_1} f(x_1, \ldots, x_n) ~dV + \idotsint_{\Omega_2} f(x_1, \ldots, x_n) ~dV$$

\end{trditev}
\vspace{0.5cm}

\begin{trditev}

Naj bosta $f_1$ in $f_2$ \textit{zvezni} funkciji na $\Omega$ ter $c_1$ in $c_2$ poljubni konstanti. Potem velja
$$\idotsint_{\Omega} (c_1 f_1 + c_2 f_2) ~dV = c_1 \idotsint_{\Omega} f_1 ~dV + c_2 \idotsint_{\Omega} f_2 ~dV.$$

\end{trditev}
\vspace{0.5cm}

\begin{trditev}

Če je $f \leq g$, potem je $\mathlarger{\iint_\Omega f ~dp \leq \iint_\Omega g ~dp}$ in podobno za večkratne integrale. Če je torej funkcija $f$ \textit{omejena} na mnoćici $\Omega$ navzgor s konstanto $M$, navzdol pa s konstanto $m$, potem velja
$$m p_\Omega \leq \iint_\Omega f dp \leq M p_\Omega,$$
kjer je $p_\Omega$ ploščina množice $\Omega$.

\end{trditev}
\vspace{0.5cm}

\begin{trditev}

$$\left| \iint_\Omega f ~dp \right| \leq \iint_\Omega |f| ~dp$$

\end{trditev}
\vspace{0.5cm}

\begin{trditev}

Naj bo $\Omega$ \textit{kompaktna} množica v $\mathbb{R}^2$, katere rob sestoji iz končno mnogo krivulj oblike $\vec:[a, b] \rightarrow \mathbb{R}^2$ za kake \textit{zvezno odvedljive} funkcije $\vec{r}$ in kake intervale $[a, b]$. Izberimo pravokotnik $P$, ki vsebuje $\Omega$, in naj bo $D$ poljubna delitev tega pravokotnika s premicami, vzporednimi koordinatnima osema. V vsakem od tistih delilnih pravokotnikov $P_k$ delitve $D$, ki sekajo $\Omega$, izberemo točko $\vec{r}_k \in P_k \cap \Omega$, označimo z $\Delta_k p$ ploščino pravokotnika $P_k$ in tvorimo Riemannovo vsoto
$$S_D(f) = \sum_k f(\vec{r}_k) \Delta_k p,$$
kjer teče indeks le po tistih delilnih pravokotnikih $P_k$, ki sekajo $\Omega$. Za vsako zvezno funkcijo $f$ na $\Omega$ je integral $\mathlarger{\iint_\Omega f(\vec{r}) ~dp}$ enak limiti vsot $S_D(f)$, ko gredo velikosti vseh delilnih pravokotnikov (torej največja diagonala vseh delilnih pravokotnikov) proti $0$. Natančneje, za $\forall \varepsilon > 0 ~\exists \delta > 0$, da je
$$\left| S_D(f) - \iint_\Omega f(\vec{r}) ~dp \right| < \varepsilon,$$
če je maksimalna diagonalna delilnih pravokotnikov $P_k$ manjša od $\delta$.

\end{trditev}
\vspace{0.5cm}

\begin{posledica}

Naj bo $\Omega$ podana kot
\begin{equation} \label{eq:6}
	\Omega = \{ (x, y) \in \mathbb{R}^2 \mid g_1(x) \leq y \leq g_2(x), ~a \leq x \leq b \},
\end{equation}
kjer sta $g_1$ in $g_2$ \textit{zvezni} funkciji na intervalu $[a, b]$ in $g_1(x) \leq g_2(x)$ za $\forall x \in [a, b]$. Naj bosta $M_1$ in $M_2$ taki števili, da pravokotnik $P = [a, b] \times [M_1, M_2]$ vsebuje množico $\Delta$ (torej $M_1 \leq g_1(x) \leq g_2(x) \leq M_2$ za $\forall x \in [a, b]$). Po definiciji imamo potem za vsako \textit{zvezno} funkcijo $f: \Omega \rightarrow \mathbb{R}$:
$$\iint_\Omega f(x, y) ~dp = \iint_P f_P(x, y) ~dp,$$
kjer je $f_P$ funkcija na $P$, definirana z 
$$f_P(x, y) = 
\begin{cases}
	f(x, y) &; ~(x, y) \in \Omega \\
	0 &; ~(x, y) \in P \setminus \Omega.
\end{cases}$$
Po zgornjem izreku pa je
$$\iint_P f_P(x, y) ~dp = \int_a^b \left( \int_{M_1}^{M_2} f(x, y) ~dy \right) dx = \int_a^b \left( \int_{g_1(x)}^{g_2(x)} f(x, y) ~dy \right) dx,$$
kjer smo upoštevali, da je funkcija $f_P$ enaka $0$ izven $\Omega$ in zato $\mathlarger{\int_{M_1}^{M_2} f_P(x, y) ~dy = \int_{g_1(x)}^{g_2(x)} ~dy}$. Torej velja naslednja trditev:

\end{posledica}
\vspace{0.5cm}

\begin{trditev}

Za vsako \textit{zvezno} funkcijo $f$ na množici $\Omega$, definirani kot (\ref{eq:6}), velja
$$\iint_\Omega f(x, y) ~dp = \int_a^b \left( \int_{g_1(x)}^{g_2(x)} f(x, y) ~dy \right) dx.$$

\end{trditev}
\vspace{0.5cm}

\begin{trditev}

Za območja $\Omega$, podana kot \\$\Omega = \{ (x, y, z) \in \mathbb{R}^3 \mid g_1(x, y) \leq z \leq g_2(x, y), ~(x, y) \in \Lambda \}$, in (skoraj povsod) \textit{zvezne} funkcije $f$ na njih velja
$$\iiint_\Omega f(x, y, z) ~dV = \iint_\Lambda \left( \int_{g_1(x)}^{g_2(x)} f(x, y, z) ~dz \right) dp.$$

\end{trditev}
\vspace{0.5cm}

% *************************************************************************************************

\subsection{Cilindrične ali valjne koordinate}
\vspace{0.5cm}

\begin{definicija}

Lega točke $(x, y, z)$ v prostoru $\mathbb{R}$ je določena s koordinato $z$ in polarnima koordinatama $r, \varphi$ njene projekcije $(x, y, 0)$ na ravnino $x,y$. Trojko $\varphi, r, z$ imenujemo \textit{cilindrične} ali \textit{valjne} koordinatne točke. S kartezičnimi koordinatami so povezane prek enakosti
$$x = r \cos{\varphi}, ~~y = r \sin{\varphi}, ~~z = z.$$
Pri tem lahko $r$ zavzame vse nenegativne vrednosti, $z$ vse realne vrednosti, $\varphi$ pa na intervalu $[0, 2\pi)$. Za dano točko $T$ pomeni $r$ njeno razdaljo od osi $z$, ki je enaka razdalji projekcije točke $T$ na ravnino $x,y$ od koordinatnega izhodišča.

\end{definicija}
\vspace{0.5cm}

\begin{posledica}[Koordinatne ploskve]
~\\
\begin{itemize}
	\item Ploskve $z = \text{konstanta}$ so ravnine, vzporedne z ravnino $x,y$
	\item Ploskve $r = \text{konstanta}$ so neskončni valji, katerih os je os $z$
	\item Ploskve $\varphi = \text{konstanta}$ pa so polravnine
\end{itemize}

\end{posledica}
\vspace{0.5cm}

% *************************************************************************************************

\subsection{Sferične koordinate}
\vspace{0.5cm}

\begin{definicija}

\textit{Sferične} ali \textit{krogelne koordinate} točke $T(x, y, z)$ so:
\begin{itemize}
	\item $R = \sqrt{x^2 + y^2 + z^2}$ razdalja od izhodišča
	\item $\theta$ kot, ki ga vektor $\vec{0T}$ oklepa s pozitivnim poltrakom osi $z$
	\item $\varphi$ kot, ki ga pravokotna projekcija vektorja $\vec{0T}$ na ravnino $x,y$ oklepa s pozitivnim poltrakom osi $x$
\end{itemize} 
Naj bo kot doslej r, razdalja $T$ od osi $z$. Potem je $r = R \sin{\theta}$ in
$$x = R \sin{\theta} \cos{\varphi}, ~~y = R \sin{\theta} \sin{\varphi}, ~~z = R \cos{\theta}.$$
Tukaj lahko zavzame kot $\theta$ vrednosti na intervalu $[0, \pi]$ ($0$ je na pozitivnem, $\pi$ pa na negativnem poltraku osi $z$), kot $\varphi$ pa vrednosti na intervalu $[0, 2\pi)$. \\

\noindent Volumni element v sferičnih koordinatah je
$$dV = R^2 \sin{\theta} ~dR ~d\theta ~d\varphi.$$
Od tod sledi, da lahko trojni integral po telesu $\Omega$, ki je opisano kot
$$\Omega = \{ (x, y, z) \in \mathbb{R}^3: ~g_1(\varphi, \theta) \leq R \leq g_2(\varphi, \theta), ~(\varphi, \theta) \in \Lambda \},$$
kjer sta $g_1 \leq g_2$ \textit{zvezni} funkciji na množici $\Lambda \subset \mathbb{R}^2$, izrazimo kot
$$\iiint_{\Omega} f(x, y, z) ~dV = \iint_{\Lambda} \left( \int_{g_1(\varphi, \theta)}^{g_2(\varphi, \theta)} f(R \sin{\theta} \cos{\varphi}, ~R \sin{\theta} \sin{\varphi}, ~R \cos{\theta}) R^2 \sin{\theta} ~dR \right) d\theta ~d\varphi.$$

\end{definicija}
\vspace{0.5cm}

\begin{posledica}[Koordinatne ploskve]
~\\
\begin{itemize}
	\item Ploskve $R = \text{konstanta}$ so sfere
	\item Ploskve $\theta = \text{konstanta}$ so stožci
	\item Plosvke $\varphi = \text{konstanta}$ so polravnine
\end{itemize}

\end{posledica}
\vspace{0.5cm}


% *************************************************************************************************

\subsection{Splošne koordinate}
\vspace{0.5cm}

\begin{definicija}

Naj bo $V$ \textit{odprta} podmožica v ravnini. Vlogo splošnih koordinat na $V$ lahko igra vsak tak par funkcij
$$u = u(x, y), ~~v = v(x, y)$$
na $V$, da iz $(x, y) \neq (x_1, y_2)$ sledi  $(u(x, y), v(x, y)) \neq (u(x_1, y_1), v(x_1, y_1))$, kar pomeni, da je točka $(x, y)$ enolično določena s parom $(u(x, y), v(x, y))$. Drugače povedano, vektorska funkcija 
$$F: V \rightarrow \mathbb{R}^2, ~F(x, y) = (u(x, y), v(x, y))$$
mora biti \textit{injektivna}. Zavoljo diferencialnega računa  predpostavimo, da sta funkciji $u$ in $v$ \textit{zvezno odvedljivi}. Iz \textit{izreka o inverzni preslikavi} vemo, da potem obrnljivost Jacobijeve matrike $F'(x, y)$ preslikave $F$ zagotavlja \textit{injektivnost} preslikave $F$ v okolici točke $(x, y)$, ne pa na celem definicijskem območju $V$, zato jo je treba posebej privzeti. Tedaj je pri pogoju, da je $F'(x, y)$ \textit{obrnljiva} matrika za $\forall (x, y) \in V$, preslikava $F$ dejansko \textit{bijekcija} na odprto množico $U := F(V)$, inverzna preslikava
$$G := F^{-1}: U \rightarrow V$$
pa je tudi \textit{zvezno odvedljiva} in 
$$G'(\vec{q}) = (F'(G(\vec{q})))^{-1} ~\text{za}~ \forall \vec{q} \in U.$$
Pri fiksnih $u_0$ in $v_0$ imenujemo krivulje $u = (x, y)$ in $v = (x, y)$ \textit{koordinatne} krivulje.

\end{definicija}
\vspace{0.5cm}

\begin{izrek}

Naj bo $G: U \rightarrow V$ taka \textit{zvezno odvedljiva bijekcija}, kjer sta $U$ in $V$ \textit{odprti} podmnožici v $\mathbb{R}$, da je $\det{G'(\vec{r})} \neq 0$ za $\forall \vec{r} \in U$. Označimo $\vec{r} = (u, v)$ in $G(u, v) = (x(u, v), y(u, v))$. Naj bo $\Omega$ \textit{kompaktna} podmnožica v $V$, katere rob naj sestoji iz končno mnogo \textit{zvezno odvedljivih} krivulj (oz. naj ima mero $0$), $f$ pa naj bo \textit{zvezna} funkcija na $V$ (razen morda na množici z mero $0$). Potem je
$$\iint_{\Omega} f(x, y) ~dx ~dy ~=~ \iint_{G^{-1}(\Omega)} f(x(u, v), y(u, v)) \left| \frac{\partial (x, y)}{\partial (u, v)}(u, v) \right| du ~dv,$$
kjer je 
$$\frac{\partial(x, y)}{\partial (u, v)}(u, v) ~=~ \mathlarger{\det{G'(u, v)} ~=~ \det{\begin{bmatrix}
\frac{\partial x}{\partial u}(u, v) & \frac{\partial x}{\partial v}(u, v) \\
\frac{\partial y}{\partial u}(u, v) & \frac{\partial y}{\partial v}(u, v)
\end{bmatrix}}}.$$

\end{izrek}
\vspace{0.5cm}

\begin{lema}

Naj bo $L: \mathbb{R}^2 \rightarrow \mathbb{R}^2$ \textit{obrnljiva linearna} preslikava in $\Lambda$ paralelogram. Potem med ploščinama paralelograma $\Lambda$ in $L(\Lambda)$ velja zveza
\begin{equation} \label{eq:7}
	p_{L(\Lambda)} = |\det{L}|_{p_{\Lambda}}.
\end{equation}
Enaka povezava velja tudi za vsako \textit{kompaktno} podmnožico $\Lambda$ v $\mathbb{R}^2$ oziroma za vsako \textit{ravninsko} podmnožico, za katero je ploščina definirana.

\end{lema}
\vspace{0.5cm}

\begin{lema}

Naj bo $G: U \rightarrow \mathbb{R}^2$ \textit{zvezno odvedljiva injektivna} preslikava s \textit{povsod obrbljivim} odvodom $G'(u, v)$, definirana na \textit{odprti} množici $U$, $K$ \textit{kompaktna} podmnožica v $U$, $\Lambda = \{ (u, v) \in \mathbb{R}^2; ~|u - a| \leq h, ~|v - b| \leq h \}$ pa kvadrat s središčem $(a, b)$ in stranico dolžine $2h$, vsebovan v $K$. Označimo
$$L ~=~ \mathlarger{\det{G'(a, b)} ~=~ \det{\begin{bmatrix}
\frac{\partial x}{\partial u}(a, b) & \frac{\partial x}{\partial v}(a, b) \\
\frac{\partial y}{\partial u}(a, b) & \frac{\partial y}{\partial v}(a, b)
\end{bmatrix}}}$$
in naj bo $A$ preslikava, definirana z $A(u, v) = G(a, b) + L(u-a,v-b)$. Potem za $\forall \varepsilon > 0 ~~\exists \delta > 0$ (neodvisen od izbire kvadrata), da za ploščini likov $G(\Lambda)$ in $A(\Lambda)$, ko je $h < \delta$, velja
$$|p_{G(\Lambda)} - p_{A(\Lambda)}| < \varepsilon p_{\Lambda}.$$
Ker se preslikavi $A$ in $L$ razlikujeta le za translacijo, lahko v tej oceni nadomestimo $A$ z $L$.

\end{lema}
\vspace{0.5cm}

% *************************************************************************************************

\pagebreak

% #################################################################################################

\section{PLOSKOVNI IN KRIVULJNI INTEGRAL}
\vspace{0.5cm}

% *************************************************************************************************

\subsection{Površina ploskve in ploskovni integral skalarne funkcije}
\vspace{0.5cm}

\begin{definicija}

Naj bo $\Lambda$ pravokotnik s središčem $(u, v) \in \Omega$ in stranicama $du$, $dv$, ki naj bosta vzporedni koordinatnima osema in je $L$ Jacobijeva matrika preslikave $\vec{r} = \vec{r}(u, v)$, torej
$$L = \begin{bmatrix}
\frac{\partial}{\partial u}\vec{r}(u, v) & \frac{\partial}{\partial v}\vec{r}(u, v)
\end{bmatrix}.$$
Ploščina paralelograma $L(\Lambda)$ je tako 
$$p_{L(\Lambda)} ~=~ \|du L(1,0) \times dv L(0, 1)\| ~=~ du~dv~\|\frac{\partial}{\partial u}\vec{r}(u, v) \times \frac{\partial}{\partial v}\vec{r}(u, v)\|.$$
Celotno površino ploskve lahko izračunamo tako, da seštejemo ploščine takih paralelogramov in limitiramo njihove velikosti proti $0$, s čimer preide vsota v integral
$$p ~=~ \iint_{\Omega} \|\frac{\partial}{\partial u}\vec{r}(u, v) \times \frac{\partial}{\partial v}\vec{r}(u, v)||~du~dv.$$
Z upoštevanjem Lagranjeove identitete $\|\vec{a} \times \vec{b}\| = \sqrt{\|\vec{a}\|^2 \|\vec{b}\|^2 - (\vec{a} \cdot \vec{b})^2}$, lahko nekoliko poenostavimo formulo:
$$E(u, v) = \|\frac{\partial}{\partial u}\vec{r}\|^2, ~~F(u, v) = \frac{\partial}{\partial u}\vec{r} \cdot \frac{\partial}{\partial v}\vec{r}, ~~G(u, v) = \|\frac{\partial}{\partial v}\vec{r}\|^2$$
$$p ~=~ \iint_{\Omega} \sqrt{EG - F^2}~du~dv.$$
Označimo $\vec{q} = (u, v)$. Jacobijeva matrika $\dot{\vec{r}}(\vec{q})$ je 
$$[\dot{\vec{r}}(\vec{q})]^T[\dot{\vec{r}}(\vec{q})] ~=~ \begin{bmatrix}
E(u, v) & F(u, v) \\
F(u, v) & G(u, v)
\end{bmatrix},$$
od kjer dobimo determinanto
$$\det{[\dot{\vec{r}}(\vec{q})]^T[\dot{\vec{r}}(\vec{q})]} ~=~ E(u, v)G(u, v) - F(u, v)^2.$$ 
\vspace{0.5cm}

\noindent Zamenjava spremenljivk:
\begin{align*}
\iint_{\Lambda} \sqrt{E(s, t)G(s, t) - F(s, t)^2}~ds~dt ~&=~ \iint_{\Lambda} \sqrt{E(s, t)G(s, t) - F(s, t)^2}\left|\frac{\partial(u, v)}{\partial(s, v)}\right|~ds~dt \\
&=~ \iint_{\Omega} \sqrt{E(u, v)G(u, v) - F(u, v)^2}~du~dv.
\end{align*}

\end{definicija}
\vspace{0.5cm}

\begin{trditev}

Definicija površine ploskve je neodvisna od parametrizacije.

\end{trditev}
\vspace{0.5cm}

\begin{definicija}

Naj bo $f$ funkcija, definirana na ploskvi $\mathcal{P}$ z enačbo \\$\vec{r} = \vec{r}(u, v)$, $(u, v) \in \Omega \subseteq \mathbb{R}^2$. \textit{Ploskovni integral} te funkcije po ploskvi $\mathcal{P}$ je definirana kot 
$$\iint_{\mathcal{P}} f~dp ~=~ \iint_{\Omega} f(\vec{r}(u, v))\sqrt{E(u, v)G(u, v) - F(u, v)^2}~du~dv.$$
Lahko bi pokazali, da je ploskovni integral neodvisen od parametrizacije ploskve.

\end{definicija}
\vspace{0.5cm}

% *************************************************************************************************

\subsection{Krivuljni integral}
\vspace{0.5cm}

\subsubsection{Krivuljni integral skalarne funkcije}

\begin{definicija}

Naj bo $f$ (\textit{zvezna}) funkcija na krivulji $\gamma$ z enačbo $\vec{r} = \vec{r}(t)$, $a \leq t \leq b$. \textit{Krivuljni integral} funkcije $f$ po krivulji $\gamma$, je definiran kot
$$\int_{\gamma} f(\vec{r})~ds ~=~ \int_{a}^{b} f(\vec{r}(t))\|\dot{\vec{r}}(t)\|~dt.$$

\end{definicija}
\vspace{0.5cm}

\subsubsection{Krivuljni integral vektorske funkcije}

\begin{definicija}

\textit{Krivuljni integral} je definiran kot
$$\int_{\gamma} \vec{F} \cdot d\vec{r} ~=~ \int_{a}^{b} \vec{F}(\vec{r}(t)) \cdot \dot{\vec{r}}(t) ~dt.$$

\end{definicija}
\vspace{0.5cm}

\begin{trditev}

Če je $t = t(\tau)$ ($\alpha \leq \tau \leq \beta$) \textit{naraščajoča zvezno odvedljiva} funkcija novega parametra $\tau$ in $t$ preteče interval $[a, b]$, ko $\tau$ preteče interval $[\alpha, \beta]$, potem je
$$\int_a^b \vec{F}(\vec{r}(t)) \cdot \frac{d}{dt}\vec{r}(t) dt ~=~ \int_{\alpha}^{\beta} \vec{F}(\vec{r}(t(\tau))) \cdot \frac{d}{dt}\vec{r}(t(\tau)) d\tau,$$
kar pomeni, da je krivuljni integral enak za vse parametrizacije dane krivulje z enako orientacijo. 

\noindent Če je pa $t = t(\tau)$ \textit{padajoča} funkcija in $t$ preteče interval $[a, b]$, ko $\tau$ teče od $\beta$ do $\alpha$, potem je
$$\int_a^b \vec{F}(\vec{r}(t)) \cdot \frac{d}{dt}\vec{r}(t) dt ~=~ -\int_{\alpha}^{\beta} \vec{F}(\vec{r}(t(\tau))) \cdot \frac{d}{dt}\vec{r}(t(\tau)) d\tau,$$
kar pomeni, da pri spremembi orientacije krivulje, krivuljni integral spremeni predznak.

\end{trditev}
\vspace{0.5cm}

\begin{trditev}
~
\begin{enumerate}
\item[(i)] $$\int_{\gamma \dotplus \lambda} \vec{F} \cdot d\vec{r} ~=~ \int_{\gamma} \vec{F} \cdot d\vec{r} + \int_{\lambda} \vec{F} \cdot d\vec{r},$$
kjer je $\vec{F}$ \textit{vektorsko polje} na $\gamma \dotplus \lambda$
\item[(ii)] $$\int_{\gamma^-} \vec{F} \cdot d\vec{r} ~=~ -\int_{\gamma} \vec{F} \cdot d\vec{r}$$
\end{enumerate}
Krivuljo $\gamma$, katere končna točka se ujema z začetno, imenujemo \textit{sklenjena}.

\end{trditev}
\vspace{0.5cm}

\begin{trditev}

Naj bo $\vec{F}$ \textit{vektorsko polje} na $U$. Velja
$$\int_{\gamma} \vec{F} \cdot d\vec{r} ~=~ 0$$
za vsako sklenjeno krivuljo $\gamma$ v $U$ natanko tedaj, ko je 
$$\int_{\gamma_1} \vec{F} \cdot d\vec{r} ~=~ \int_{\gamma_2} \vec{F} \cdot d\vec{r}$$
za vsaki krivulji $\gamma_1$ in $\gamma_2$ v $U$, ki imata isto začetno in isto končno točko.

\end{trditev}
\vspace{0.5cm}

\subsubsection{Krivuljni integral potencialnega polja}

\begin{definicija}

Vektorsko polje $\vec{F}: U \rightarrow \mathbb{R}^3 ~(\mathbb{R}^2)$ je \textit{potencialno}, če obstaja taka funkcija $u: U \rightarrow \mathbb{R}^3 ~(\mathbb{R}^2)$, imenovana \textit{potencial polja} $\vec{F}$, da je
$$ \vec{F} ~=~ \vec{\nabla} u,$$
kjer je $\vec{\nabla} u$ gradient funkcije $u$.

\end{definicija}
\vspace{0.5cm}

\begin{trditev}

\textit{Zvezno odvedljivo} vektorsko polje $\vec{F} = (M, N)$ je potencialno le, če velja
$$\frac{\partial M}{\partial y} ~=~ \frac{\partial N}{\partial x}.$$

\end{trditev}
\vspace{0.5cm}

\begin{trditev}

Krivuljni integral \textit{potencialnega} vektorskega polja $\vec{\nabla} u$ po katerikoli krivulji $\gamma$ v definicijskem območju polja je
$$\int_{\gamma} \vec{\nabla} u \cdot d\vec{r} ~=~ u(\vec{r}(b)) - u(\vec{r}(a)),$$
kjer sta $\vec{r}(a)$ in $\vec{r}(b)$ začetna in končna točka krivulje $\gamma$. Torej je krivuljni integral danega potencialnega polja neodvisen od poteka krivulje, odvisen je le od njene začetne in končne točke.

\end{trditev}
\vspace{0.5cm}

\begin{trditev}

Vektorsko polje $\vec{F}$ na območju $U$ je potencialno natanko tedaj, ko je za vsaki dve točki $\vec{r}_0$ in $\vec{r}$ iz $U$ krivuljni integral $\mathlarger{\int_{\gamma} \vec{F} \cdot d\vec{r}}$ enak za vse krivulje $\gamma$ v $U$ z začetno točko $\vec{r}_0$ in končno točko $\vec{r}$.

\end{trditev}
\vspace{0.5cm}

\begin{posledica}

Vektorsko polje $\vec{F}$ na $U$ je potencialno natanko tedaj, ko je 
$$\int_{\gamma} \vec{F} \cdot d\vec{r} ~=~ 0$$
za vsako sklenjeno krivuljo $\gamma$ v $U$.

\end{posledica}
\vspace{0.5cm}

\subsubsection{Povezava med krivuljnim in dvojnim integralom}

\begin{izrek}

Naj bo $\Omega$ kompaktna ravninska množica, katere rob $\partial \Omega$ sestoji iz končno mnogo sklenjenih zvezno odvedljivih krivulj, parametriziranih na kompaktnih interavlih in usmerjenih tako, da je $\Omega$ na njihovi levi. Za vsako \textit{zvezno odvedljivo} vektorsko polje $\vec{F} = (M, N)$, definirano na kaki okolici množice $\Omega$, velja Greenova formula
$$\int_{\partial \Omega} (M~dx + N~dy) ~=~ \iint_{\Omega} \left( \frac{\partial N}{\partial x} - \frac{\partial M}{\partial y} \right)~dp.$$
Ravninsko območje $U$ brez lukenj imenujemo \textit{enostavno povezano}. Natančneje: $U$ je enostavno povezano, če je njegov komplement v razširjeni ravnini (torej v $\mathbb{R}^2 \cup \{\infty\}$ povezana množica, se pravi je iz enega kosa).

\end{izrek}
\vspace{0.5cm}

\begin{posledica}

Na \textit{enostavnem povezanem} območju $U$ je pogoj 
$$\frac{\partial M}{\partial y} ~=~ \frac{\partial N}{\partial x}$$
potreben in zadosten za potencialnost vektorskega polja $\vec{F} = (M, N)$.

\end{posledica}
\vspace{0.5cm}

% *************************************************************************************************

\pagebreak

% #################################################################################################

\section{DIFERENCIALNE ENAČBE}
\vspace{0.5cm}

\begin{definicija}

\textit{Splošna diferencialna enačba} reda $n$ je enačba oblike 
$$F(x, y, y', y'', \ldots, y^{(n)}) = 0,$$
kjer je $F$ dana funkcija $n+2$ spremenljivk definirana na kakem območju v $\mathbb{R}^{n+1}$, $y$ neznana funkcija spremenljivke $x$, $y^{(k)}$ ($k = 1, \ldots, n$) pa njeni odvodi. \textit{Red enačbe} je red najvišjega odvoda, ki nastopa v enačbi.

\end{definicija}
\vspace{0.5cm}

% *************************************************************************************************

\subsection{Enačbe 1. reda}
\vspace{0.5cm}

\begin{definicija}[Enačba z ločljivima spremenljivkama]

To je enačba oblike
$$g(y) y' = f(x),$$
kjer sta $f$ in $g$ \textit{zvezni} funkciji. V tem primeru obe strani lahko integriramo in dobimo
$$\int g(y) dy = \int f(x) dx ~~~\Rightarrow~~~ G(y) = F(x) + C,$$
kjer sta $F$ in $G$ \textit{primitivni funkciji} $f$ in $g$ ter je $C$ \textit{integracijska konstanta}. Če se da izraziti $y = y(x)$ smo dobili rešitev, sicer rečemo, da zgornja enačba predstavlja rešitev v \textit{implicitni} obliki.

\end{definicija}
\vspace{0.5cm}

\begin{definicija}[Enačba oblike $y' = f(\frac{y}{x})$]

Naj bo $f$ zvezna funkcija. Z novo neznanko $v = \cfrac{y}{x}$, torej $y = xv$ in $y' = xv'$, kar enačbo preoblikuje v 
$$xv' = f(v) - v,$$
kjer sta spremenljivki $x$ in $v$ ločljivi.

\end{definicija}
\vspace{0.5cm}

\begin{definicija}[Linearna enačba 1. reda]

To je vsaka enačba, ki se jo da preoblikovati v
$$y' = py + q,$$
kjer sta $p$ in $q$ \textit{zvezni} funkciji na kakem intervalu $I$ (lahko tudi poltrak ali cela realna os). Če je $q \equiv 0$ imenujemo enačbo \textit{homogena}. Tedaj sta spremenljivki ločljivi. Ko $q \not\equiv 0$, najprej enačbo rešimo homogeno enačbo in nato rešitev vstavimo v enačbo (\textit{variacija konstante}). Dobimo
$$y' = C' e^P + C e^P P,$$
kjer je $P' = p$. Originalno enačbo tako preoblikujemo v
$$C' e^P = q.$$
Od tod izračunamo $\mathlarger{C = \int_a^x e^{-P(t)} q(t) dt + K}$, kjer je $K$ konstanta in sledi 
$$y(x) ~=~ C e^{P(x)} ~=~ e^{P(x)} \int_a^x e^{-P(x)} q(t) dt + K e^{P(x)}.$$

\end{definicija}
\vspace{0.5cm}

\begin{definicija}[Bernoullijeva enačba]

To je enačba oblike
$$y' = py + qy^n,$$
kjer sta $p$ in $q$ \textit{zvezni} funkciji in $n \in \mathbb{R} \setminus \{0,1\}$ \textit{realna} konstanta. Za $n > 0$ je ena rešitev te enačbe $y \equiv 0$. Pri deljenju enačbe z $y^n$ vidimo, da jo vpeljava nove neznanke $v = y^{1-n}$ spremeni v \textit{linearno}. Tedaj $y = v^{\frac{1}{1-n}}$ in $y' = \frac{1}{1-n} v^{\frac{1}{1-n}-1} v'$, enačba pa se preoblikuje v
$$\frac{1}{1-n} v' = pv + q,$$
ki je linearna. Pri deljenju z $v^{\frac{1}{1-n}}$ izgubimo rešitve $v$, ki so za $n>0$ enake $0$. V vsaki ničli $x_0$ funkcije $v$ je tudi $y(x_0) = 0$ in $y'(x_0) = 0$, če je $n \in (0,1)$, zato lahko v točki $(x_0,0)$ združimo rešitvi $y \equiv 0$ in $y = v^{\frac{1}{1-n}}$.

\end{definicija}
\vspace{0.5cm}

\begin{definicija}[Eksaktna enačba]

Diferencialno enačbo $y' = f(x,y)$ lahko zapišemo tudi kot $f(x,y)dx - dy = 0$. Splošnejšo enačbo take oblike
$$\omega := f(x,y)dx + g(x,y)dy = 0,$$
znamo rešiti, če je izraz $\omega$ \textit{totalni diferencial} kake funkcije $u$, torej
$$\omega = du = \frac{\partial u}{\partial x}dx + \frac{\partial u}{\partial y}dy.$$
Tedaj $\omega = du = 0$ pomeni (na povezanem območju v ravnini), da je funkcija $u$ konstantna, torej $u(x, y) = C$, kar imamo lahko za \textit{implicitno} podano rešitev. \\
\noindent Pogoj, da je $\omega$ totalni diferencialn oz. da je $f = \cfrac{\partial u}{\partial x}$ in $g = \cfrac{\partial u}{\partial y}$ za kako funkcijo $u$, je enakost
$$\frac{\partial f}{\partial y} ~=~ \frac{\partial^2 u}{\partial y \partial x} ~=~ \frac{\partial^2 u}{\partial x \partial y} ~=~ \frac{\partial g}{\partial x}.$$ 
Na splošno $\omega$ ni totalni diferencial, lahko pa morda najdemo tako funkcijo $\mu$ brez ničel, da je $\mu \omega$ totalni diferencial funkcije $u$. Tak $\mu$ imenujemo \textit{integrirajoči množitelj}. Ker je $\omega = 0$ ekvivalentna enačbi $du = \mu \omega = 0$, je $u(x, y) = C$ spet implicitno podana rešitev. \\
\noindent Pogoj, da je $\mu \omega$ totalni diferencial kake funkcije je
$$\frac{\partial (\mu f)}{\partial y} ~=~ \frac{\partial (\mu g)}{\partial x} ~~\text{oz.}~~ \frac{1}{\mu} \left( g \frac{\partial \mu}{\partial x} - f \frac{\partial \mu}{\partial y} \right) ~=~ \frac{\partial f}{\partial y} - \frac{\partial g}{\partial x}.$$ 
Če $\exists \mu = \mu(x)$, torej $\frac{\partial \mu}{\partial y} = 0$, se enakost poenostavi ter dobimo (kjer $g$ ni $0$)
$$\frac{d(\ln \mu)}{dx} ~=~ \frac{\frac{\partial f}{\partial y} - \frac{\partial g}{\partial x}}{g}.$$
Če je izraz na desni odvisen le od $x$, potem je 
$$\mu = e^{\int \frac{\frac{\partial f}{\partial y} - \frac{\partial g}{\partial x}}{g} dx}.$$
integrirajoči množitelj. Podobno lahko izpeljemo, če $\exists \mu = \mu(y)$

\end{definicija}
\vspace{0.5cm}

% *************************************************************************************************

\subsection{Homogene linearne diferencialne enačbe 2. reda}
\vspace{0.5cm}

\begin{definicija}

\textit{Splošna linearna diferencialna} enačba 2. reda je enačba oblike
$$f(x) y'' + g(x) y' + h(x) y ~=~ d(x),$$
kjer so $f$, $g$, $h$ in $d$ \textit{zvezne} funkcije na kakem intervalu $I$, $y$ pa neznana, \textit{dvakrat zvezno odvedljiva} funkcija. Če je desna stran $d \equiv 0$, imenujemo enačbo \textit{homogena}. Kadar $f$ nima ničel, dobimo po deljenju z $f$ ekvivalentno enačbo oblike
$$y'' + p(x) y' + q(x) y = r(x),$$
kjer so $p$, $q$ in $r$ \textit{zvezne} funkcije. Vse funkcije tukaj imajo lahko vrednosti v $\mathbb{C}$. Zgornje enačbe ne moremo vedno rešiti \textit{eksplicitno} z znanimi funkcijami, rešitve pa vedno obstajajo in so določene enolično pri začetnih pogojih.

\end{definicija}
\vspace{0.5cm}

\begin{izrek}

Če so $p$, $q$ in $r$ \textit{zvezne} funkcije na intervalu $I$, potem za $\forall x_0 \in I$ in poljubni konstanti $y_0, \tilde{y}_0 \in \mathbb{C}$ obstaja natanko ena \textit{dvakrat zvezno odvedljiva} funkcija $y: I \rightarrow \mathbb{C}$, ki zadošča enačbi $y'' + p(x) y' + q(x) y = r(x)$ in začetnima pogojema 
$$y(x_0) = y_0, ~~~y'(x_0) = \tilde{y}_0.$$
Če so pri tem $p$, $q$, $r$ \textit{realne} funkcije in $y_0, \tilde{y}_0 \in \mathbb{R}$, potem je tudi rešitev $y$ \textit{realna} funkcija.

\end{izrek}
\vspace{0.5cm}

\begin{definicija}[Linearna neodvisnost]

Funkciji $y_1$ in $y_2$ sta \textit{linearno neodvisni}, če nobena njuna \textit{netrivialna linearna} kombinacija $c_1 y_1 + c_2 y_2$ ($c_1, c_2 \in \mathbb{C}$ sta konstanti, $(c_1, c_2) \neq (0,0)$) ni identično enaka $0$. To pomeni, da nobena od obeh funkcij ni konstanten večkratnik druge.

\end{definicija}
\vspace{0.5cm}

\begin{definicija}

\textit{Determinanta Wronskega} funkcij $y_1, y_2$ je funkcija, definirana kot
$$W_{y_1,y_2}(x) ~=~ \begin{vmatrix}
y_1(x) & y_2(x) \\
y_1'(x) & y_2'(x)
\end{vmatrix} ~=~ y_1(x) y_2'(x) - y_1'(x) y_2(x).$$

\end{definicija}
\vspace{0.5cm}

\begin{trditev}[Liouvillova formula]

Za determinanto Wronskega $W = W_{y_1,y_2}$ dveh rešitev $y_1, y_2$ enačbe $y'' + p(x) y' + q(x) y = 0$ velja \textit{Liouvillova formula}
$$W(x) = W(x_0) e^{-\int_{x_0}^x p(t) dt}$$
za $\forall x \in I$, kjer je $x_0$ poljubna točka iz intervala $I$, nad katerim opazujemo enačbo. Torej je $W$ bodisi enaka $0$ bodisi nima nobene ničle na $I$.

\end{trditev}
\vspace{0.5cm}

\begin{trditev}

Rešitvi $y_1, y_2$ enačbe $y'' + p(x) y' + q(x) y = 0$ sta na intervalu $I$ \textit{linearno neodvisni} natanko tedaj, ko njuna determinanta Wronskega $W$ ni identično enaka $0$ na $I$, kar je natanko takrat, ko $W$ nima nobene ničle na $I$.

\end{trditev}
\vspace{0.5cm}

\begin{izrek}

Množica vseh rešitev enačbe $y'' + p(x) y' + q(x) y = 0$, je \textit{dvorazsežen vektorski prostor}. Če sta torej $y_1$ in $y_2$ linearno neodvisni rešitvi, potem lahko vsako rešitev $y$ izrazimo kot njuno \textit{linearno kombinacijo}, 
$y ~=~ c_1 y_1 + c_2 y_2,$
kjer sta $c_1, c_2$ primerni konstanti (v splošnem kompleksni, sicer pa realni, če nas zanimajo le realne rešitve in sta $y_1$ in $y_2$ realni funkciji).

\end{izrek}
\vspace{0.5cm}

\begin{trditev}

Naj bo $y_p$ \textit{partikularna} rešitev enačbe $y'' + p y' + q y = r$ (torej neka konkretna rešitev), $y$ pa poljubna nadaljna rešitev. Potem je funkcija
$$y_h := y - y_p$$
rešitev ustrezne \textit{homogene} enačbe $y'' + p y' + q y = 0$. \\
\noindent Velja tudi obratno: za vsako rešitev $y_h$ \textit{homogene} enačbe $y'' + p y+ + g y = 0$ je vsota
$$y = y_p + y_h$$
rešitev enačbe $y'' + p y' + q y = r$.

\end{trditev}
\vspace{0.5cm}

% *************************************************************************************************

\subsection{Enačbe s konstantnimi koeficienti in Eulerjeva enačba}
\vspace{0.5cm}

\begin{definicija}

Enačbo oblike 
$$x^2 y'' + p x y' + q y ~=~ 0, ~~(x > 0),$$
kjer ta $p$ in $q$ konstanti, imenujemo (homogena) \textit{Eulerjeva enačba}. To enačbo rešujemo z vplejavo nove neodvisne spremenljivke prek zveze $x = e^t$, kjer dobimo zvezi $y' = e^{-t} \dot{y}$ in $y'' = e^{-2t}(\ddot{y} - \dot{y})$, kjer je $\dot{y} = \cfrac{dy}{dt}$ in dobimo
$$\ddot{y} + (p-1) \dot{y} + q y ~=~ 0.$$

\end{definicija}

% *************************************************************************************************

\subsection{Sistemi linearnih diferencialnih enačb s konstantnimi koeficienti}
\vspace{0.5cm}

\begin{definicija}

\textit{Sistem linearnih diferencialnih enačb prvega reda}
$$\dot{x}_i ~=~ \sum_{j=1}^n a_{i,j}(t)x_j + f_i(t), ~~i = 1, \ldots, n,$$
kjer so $a_{i,j}$ in $f_i$ dane zvezne spremenljivke $t$, $x_i$ pa neznanke, lahko zapišemo kot
$$\frac{d}{dt}\vec{x} ~=~ A(t)\vec{x} + \vec{f}(t),$$
kjer smo vpeljali matrično funkcijo $A(t) := [a_{i,j}(t)]$ ter vektoski funkciji $\vec{f}(t) = (f_1(t), \ldots, f_n(t))^T$ in $\vec{x}(t) = (x_1(t), \ldots, x_n(t))^T$. Vse te funkcije so definirane na kakem intervali $(a, b)$, ki je lahko tudi poltrak ali pa cela realna os $\mathbb{R}$. \\

\noindent Kadar je $\vec{f} \equiv 0$, imenujemo sistem $\textit{homogen}$. V primerih, ko je matrika $A$ konstantna, lahko homogen sistem $\cfrac{d}{dt}\vec{x} = A\vec{x}$ rešimo elementarno. Tedaj namreč za matrično funkcijo $e^{tA} := \mathlarger{\sum_{n=0}^{\infty} \frac{t^n}{n!} A^n}$ (vrsta konvergira za $\forall t \in \mathbb{R}$) velja enakost
$$\frac{d}{dt}\left( e^{tA} \right) ~=~ A e^{tA}.$$
Zato za njen inverz $e^{-tA}$ velja
$$\frac{d}{dt}\left( e^{-tA} \vec{x} \right) ~=~ \vec{0},$$
kar pove, da je vektorska funkcija $t \mapsto e^{-tA}\vec{x}$ konstantna, recimo enaka $\vec{c}$, torej 
$$\vec{x} ~=~ e^{tA}\vec{c}.$$

\end{definicija}
\vspace{0.5cm}

\begin{trditev}

Naj bo $A$ matrika reda $2 \times 2$. Če se da $A$ diagonalizirati in sta $\lambda_1, \lambda_2$ njeni lastni vrednosti, je splošna rešitev sistema $\mathlarger{\frac{d}{dt}\vec{x} = A\vec{x}}$ oblike
$$\vec{x} ~=~ e^{\lambda_1 t}\vec{a} + e^{\lambda_2 t}\vec{b},$$
kjer sta $\vec{a}$ in $\vec{b}$ lastna vektorja matrike $A$, ki pripadata lastnima vrednostma $\lambda_1$ in $\lambda_2$ (zaporedoma). Če pa se $A$ ne da diagonalizirati, potem je splošna rešitev že omenjenega sistema oblike
$$\vec{x} = e^{\lambda t}(\vec{a} + t\vec{b}),$$
kjer je $\lambda$ lastna vrednost matrike $A$, $\vec{b}$ pripadajoči vektor, $\vec{a}$ pa tak vektor, da je $(A - \lambda I)\vec{a} = \vec{b}$.

\end{trditev}
\vspace{0.5cm}

% *************************************************************************************************

\pagebreak

% #################################################################################################

\section{METRIČNI PROSTORI}
\vspace{0.5cm}

% *************************************************************************************************

\subsection{Metrika, krogla, odprt in zaprte množice}
\vspace{0.5cm}

% -------------------------------------------------------------------------------------------------

\subsubsection{Metrika}
\vspace{0.5cm}

\begin{definicija}

\textit{Metrični prostor} je \textit{neprazna} množica $M$, opremljena s preslikavo
$$d: M \times M \rightarrow [0, \infty),$$
ki ima naslednja lastnosti za $\forall x, y, z \in M$:
\begin{enumerate}
	\item $d(x, z) \leq d(x, y) + d(y, z)$ (trikotniška neenakost)
	\item $d(y, x) = d(x, y)$
	\item $d(x, y) = 0 ~\iff~~ x = y$
\end{enumerate}
Preslikavi $d$ pravimo \textit{razdalja} ali \textit{matrika}, elemente množice $M$ pa imenujemo tudi \textit{točke}. Pogosto bomo rekli, da je metrični prostor kar par $(M, d)$.

\end{definicija}
\vspace{0.5cm}

\begin{definicija}

Naj bo $M$ metrični prostor z metriko $d$. Za $\forall a \in M$ in za $\forall r \in [0, \infty)$ imenujemo množico 
$B(a, r) ~=~ {x \in M: d(x, a) < r}$
\textit{odprta krogla} s središčem $a$ in polmerom $r$. Množico 
$\overline{B}(a, r) ~=~ {x \in M: d(x, a) \leq r}$
pa imenujemo \textit{zaprta krogla} s središčem $a$ in polmerom $r$. \\

\noindent Očitno je $B(a, r) \subseteq \overline{B}(a, r) \subseteq B(a, r + \varepsilon)$ za $\forall \varepsilon > 0$.

\end{definicija}
\vspace{0.5cm}

% -------------------------------------------------------------------------------------------------

\subsubsection{Odprte in zaprte množice}
\vspace{0.5cm}

\begin{definicija}

Podmnožica $G$ metričnega prostora $M$ je \textit{odprta}, če za $\forall a \in G$ obstaja tako realno število $r > 0$, da je $B(a, r) \subseteq G$. Podmnožica $F$ je \textit{zaprta} v $M$, če je njen komplement $F^C$ \textit{odprta} množica. Intuitivno: v ravnini (ali pa prostoru) so odprte tiste podmnožice, ki ne vsebujejo svojega roba, zaprte pa tiste, ki svoj rob vsebujejo.

\end{definicija}
\vspace{0.5cm}

\begin{posledica}

Množici $M$ in $\emptyset$ sta hratki odrti in zaprti množici. V splošnem obstajajo tudi podmnožice, ki niso niti odprte, niti zaprte.

\end{posledica}
\vspace{0.5cm}

\begin{trditev}

\textit{Odprta} krogla je odprta množica, \textit{zaprta} krogla je zaprta množica v vsakem metričnem prostoru $M$.

\end{trditev}
\vspace{0.5cm}

\begin{trditev}
~
\begin{enumerate}
	\item Unija poljubne družine \textit{odprtih} podmnožic metričnega prostora je odprta podmnožica.
	\item Presek končno mnogo \textit{odprtih} podmnožic metričnega prostora je odprta podmnožica.
\end{enumerate}

\end{trditev}
\vspace{0.5cm}

\begin{posledica}
~
\begin{enumerate}
	\item Presek poljubne družine \textit{zaprtih} podmnožic metričnega prostora je zaprta podmnožic je zaprta podmnožica.
	\item Unija končno mnogo \textit{zaprtih} podmnožic je zaprta podmnožica.
\end{enumerate}

\end{posledica}
\vspace{0.5cm}

\begin{posledica}

Podmnožica v metričnem prostoru je \textit{odprta} natanko tedaj, ko se da izraziti kot unija odprtih krogel.

\end{posledica}
\vspace{0.5cm}

% -------------------------------------------------------------------------------------------------

\subsubsection{Rob, notranjost in zaprtje}
\vspace{0.5cm}

\begin{definicija}
~
\begin{enumerate}
	\item Točka $a \in S$ je \textit{notranja točka} množice $S$, če je $B(a, r) \subseteq S$ za kak $r > 0$. Množica $\mathring{S}$ vseh notranjih točk množice $S$ imenujemo \textit{notranjost} množice $S$.
	\item Točka $a \in M$ je \textit{robna točka} za $S$, če vsaka odprta krogla $B(a, r)$ seka takò množico $S$ kot njen komplement:
	$$B(a, r) \cap S \neq \emptyset ~~\text{in}~~ B(a, r) \cap S^C \neq \emptyset ~~\text{za}~ \forall r > 0.$$
	Množico $\partial S$ vseh robnih točk imenujemo \textit{rob} množice S.
	\item Točka $a \in M$ je \textit{zunanja} za $S$, če je $B(a, r) \subseteq S^C$ za kak $r > 0$.
	\item \textit{Zaprtje} $\overline{S}$ množice $S$ je 
	$$\overline{S} ~=~ S \cup \partial S.$$
\end{enumerate}

\end{definicija}
\vspace{0.5cm}

\begin{posledica}
~
\begin{itemize}
	\item Množica je odprta $~\iff~~$ vse njene točke so notranje
	\item Notranjost vsake množice je odprta množica.
	\item Točka $a \in M$ je v zaprtju podmnožice $S$ $~\iff~~$ $B(a, r) \cap S \neq \emptyset$ za $\forall r > 0$.
	\item $\overline{S} = \mathring{S} \cup \partial S$
\end{itemize}

\end{posledica}
\vspace{0.5cm}

\begin{trditev}

Zaprtje podmnožice $S$ v metričnem prostoru $M$ je enako preseku vseh zaprtih podmnožic v $M$, ki vsebujejo $S$. Torej je podmnožica $S$ zaprta natanko tedaj, ko je $\overline{S} = S$.

\end{trditev}
\vspace{0.5cm}

\begin{definicija}

Podmnožico $S$ v metričnem prostoru $M$ imenujemo \textit{povsod gosta podmnožica}, če je $\overline{S} = M$.

\end{definicija}
\vspace{0.5cm}

\begin{definicija}

\textit{Okolica} točke $a$ v metričnem prostoru $M$ je podmnožica $G \subseteq M$, ki vsebuje kako odprto kroglo $B(a, r)$ s pozitivnim polmerom $r$.

\end{definicija}
\vspace{0.5cm}

% -------------------------------------------------------------------------------------------------

% *************************************************************************************************

\subsection{Polnost}
\vspace{0.5cm}

\begin{definicija}

Zaporedje $(a_n)$ v metričnem prostoru $(M, d)$ je \textit{konvergentno}, če obstaja taka točka $a \in M$, da za $\forall \varepsilon > 0$ obstaja tak $n_0 \in \mathbb{N}$, da je
$$d(a_n, a) < \varepsilon ~~\text{za}~~ \forall n \geq n_0.$$
Tedaj imenujemo točko $a$ \textit{limita zaporedja} $(a_n)$ in zapišemo $a = \lim_{n \rightarrow \infty} a_n$. Zaporedje, ki ni konvergentno, imenujemo \textit{divergentno}.

\end{definicija}
\vspace{0.5cm}

\begin{definicija}

Zaporedje $(a_n)$ v metričnem prostoru $(M, d)$ je \textit{Cauchyjevo}, če za $\forall \varepsilon > 0$ obstaja tak $n_0 \in \mathbb{N}$, da je 
$$d(a_n, a_m) < \varepsilon ~~\text{za}~~ \forall n, m \geq n_0.$$

\end{definicija}
\vspace{0.5cm}

\begin{definicija}

Metrični prostor $(M, d)$ je \textit{poln}, če je vsako Cauchyjevo zaporedje v njem konvergentno (z limito v $M$).

\end{definicija}
\vspace{0.5cm}

\begin{trditev}

\textit{Neprazna} podmnožica $F$ \textit{polnega} metričnega prostora $(M, d)$ je poln metrični prostor (za razdaljo $d$) natanko tedaj, ko je \textit{zaprta}.

\end{trditev}
\vspace{0.5cm}

\begin{izrek}

Za $\forall m  = 1, 2, \ldots$ je $\mathbb{R}^m$ (z običajno evklidsko normo) poln prostor. Prav tako je poln tudi prostor $\mathbb{C}^m$.

\end{izrek}
\vspace{0.5cm}

\begin{definicija}

\textit{Diameter} ali \textit{premer} podmnožice $S$ metričnega prostora $(M, d)$ je 
$$d(S) ~:=~ \sup{\{d(x,y): ~x, y \in S\}}.$$

\end{definicija}
\vspace{0.5cm}

\begin{izrek}[Cantorjev izrek o preseku]

Naj bo $(M, d)$ \textit{poln} metrični prostor, $F_1 \supseteq F_2 \supseteq F_3 \supseteq \ldots$ pa \textit{padajoče} zaporedje \textit{nepraznih zaprtih} podmnožic v $M$, katerih diametri $d(F_n)$ konvergirajo proti $0$. Potem presek 
$$F ~:=~ \cap_{n=1}^{\infty} F_n$$
vsebuje natanko eno točko.

\end{izrek}
\vspace{0.5cm}

% *************************************************************************************************

\subsection{Kompaktnost}
\vspace{0.5cm}

\begin{definicija}

Točka $s \in M$ je stekališče zaporedja $(a_n)$ v metričnem prostoru $(M, d)$, če je za $\forall r > 0$ v krogli $B(s, r)$ neskončno mnogo členov zaporedja. Pri tem štejemo vsak člen tolikokrat, kolikokrat nastopa v zaporedju.

\end{definicija}
\vspace{0.5cm}

\begin{definicija}

Metrični prostor $M$ je \textit{kompakten}, če ima vsako zaporedje iz $M$ stekališče v $M$. Podmnožica metričnega prostora $M$ je \textit{kompaktna}, če je kompaktna kot metrični prostor za metriko, ki jo podeduje iz $M$.

\end{definicija}
\vspace{0.5cm}

\begin{trditev}

\textit{Zaprta} podmnožica \textit{kompaktnega} metričnega prostora je \\kompaktna.

\end{trditev}
\vspace{0.5cm}

\begin{lema}

Vsak \textit{kompakten} metrični prostor je poln.

\end{lema}
\vspace{0.5cm}

\begin{definicija}

Podmnožica $S$ metričnega prostora $(M, d)$ je \textit{omejena}, če je vsebovana v kaki krogli, torej če obstaja kako tako realno število $r > 0$, da je $S \subseteq B(a, r)$ za kako točko $a \in M$. 

\end{definicija}
\vspace{0.5cm}

\begin{izrek}[Heine-Borel]

Podmnožica v $\mathbb{R}^n$ je \textit{kompaktna} natanko tedaj, ko je \textit{zaprta} in \textit{omejena}.

\end{izrek}
\vspace{0.5cm}

% *************************************************************************************************

\subsection{Zvezne preslikave}
\vspace{0.5cm}

% -------------------------------------------------------------------------------------------------

\subsubsection{Zveznost}
\vspace{0.5cm}

\begin{definicija}

Naj bosta $(M_1, d_1)$ in $(M_2, d_2)$ metrična prostora. Preslikava $f: M_1 \rightarrow M_2$ je \textit{zvezna} v točki $a \in M_1$, če za $\forall \varepsilon > 0$ obstaja tak $\delta > 0$, da za $\forall x  \in M_1$ iz $d_1(x, a) < \delta$ sledi
$$d_2(f(x), f(a)) < \varepsilon.$$
Z drugimi besedami, $f$ je zvezna v točki $a$, če za $\forall \varepsilon > 0$ obstaja tak $\delta > 0$, da je
$$f(B(a, \delta)) ~\subseteq~ B(f(a), \varepsilon).$$
Preslikava $f$ je \textit{zvezna}, če je zvezna v vsaki točki $a \in M_1$.

\end{definicija}
\vspace{0.5cm}

\begin{trditev}

Preslikava $f: M_1 \rightarrow M_2$ je \textit{zvezna} v točki $a \in M_1$ natanko tedaj, ko za vsako zaporedje $(a_n)_n$ v $M_1$, ki konvergira proti $a$, konvergira zaporedje $(f(a_n))_n$ proti $f(a)$.

\end{trditev}
\vspace{0.5cm}

\begin{trditev}

Preslikava $f: M_1 \rightarrow M_2$ je \textit{zvezna} natanko tedaj, ko je $f^{-1}(G)$ \textit{odprta} množica v $M_1$ za vsako \textit{odprto} podmnožico $G \subseteq M_2$.

\end{trditev}
\vspace{0.5cm}

\begin{posledica}

Preslikava $f: M_1 \rightarrow M_2$ je \textit{zvezna} natanko tedaj, ko je $f^{-1}(F)$ \textit{zaprta} množica v $M_1$ za vsako \textit{zaprto} množico $F \subseteq M_2$.

\end{posledica}
\vspace{0.5cm}

\begin{trditev}

Če je $f: M_1 \rightarrow M_2$ \textit{zvezna} preslikava, je $f(K)$ \textit{kompaktna} podmnožica v $M_2$ za vsako \textit{kompaktno} podmnožico $K \subseteq M_1$.

\end{trditev}
\vspace{0.5cm}

\begin{definicija}

Preslikava $f: M_1 \rightarrow M_2$ je \textit{omejena}, če je $f(M_1)$ \textit{omejena} podmnožica v $M_2$.

\end{definicija}
\vspace{0.5cm}

\begin{posledica}

Naj bo $f: M_1 \rightarrow M_2$ \textit{zvezna} preslikava med metričnima prostoroma. Slika $f(K)$ vsake \textit{kompaktne} podmnožice $K \subseteq M_1$ je \textit{omejena} v $M_2$.

\end{posledica}
\vspace{0.5cm}

\begin{trditev}

Kompozitum \textit{zveznih} preslikav je \textit{zvezna} preslikava.

\end{trditev}
\vspace{0.5cm}

\begin{definicija}

\textit{Bijektivno zvezno} preslikavo, katere inverz je \textit{zvezen}, imenujemo \textit{homeomorfizem}. Metrična prostora $M_1$ in $M_2$ sta \textit{homeomorfna}, če obstaja kak homeomorfizem $f: M_1 \rightarrow M_2$.

\end{definicija}
\vspace{0.5cm}

% -------------------------------------------------------------------------------------------------

\subsubsection{Enakomerna zveznost}
\vspace{0.5cm}

\begin{definicija}

Preslikavo $f: M_1 \rightarrow M_2$ med metričnima prostoroma $(M_1, d_1)$ in $(M_2, d_2)$ je \textit{enakomerno zvezna}, če za $\forall \varepsilon > 0$ obstaja tak $\delta > 0$, da za vsaki točki $x, y \in M_1$ iz $d_1(x, y) < \delta$ sledi 
$$d_2(f(x), f(y)) < \varepsilon.$$

\end{definicija}
\vspace{0.5cm}

\begin{trditev}

Če je $(M_1, d_1)$ \textit{kompakten} metrični prostor, je vsaka \textit{zvezna} preslikava $f: M_1 \rightarrow M_2$ \textit{enakomerno zvezna} za vsak metrični prostor $(M_2, d_2)$. 

\end{trditev}
\vspace{0.5cm}

% -------------------------------------------------------------------------------------------------

\subsubsection{Bonus meme: Negibne točke kontrakcij}
\vspace{0.5cm}

\begin{definicija}

Preslikava $f: M \rightarrow M$ metričnega prostora $(M, d)$ vase je \textit{skrčitev} ali \textit{kontrakcija}, če obstaja kako število $q \in [0, 1)$, da je
$$d(f(x), f(y)) ~\leq~ q \cdot d(x, y)$$
za vsaka $x, y \in M$. Očiteno je vsaka kontrakcija (enakomerno) zvezna.

\end{definicija}
\vspace{0.5cm}

\begin{definicija}

Točka $x \in M$ je \textit{negibna} ali \textit{fiksna} za preslikavo \\$f: M \rightarrow M$, če $f(x) = x$.

\end{definicija}
\vspace{0.5cm}

\begin{izrek}[Banachov izrek o skrčitev]

Za vsako kontrakcijo $f: M \rightarrow M$ \textit{polnega} metričnega prostora $(M, d)$ obstaja natanko ena negibna točka \\$a \in M$.

\end{izrek}
\vspace{0.5cm}

% -------------------------------------------------------------------------------------------------

% *************************************************************************************************

\pagebreak

% #################################################################################################

\section{KOMPLEKSNA ANALIZA}
\vspace{0.5cm}

% *************************************************************************************************

\subsection{Poti in območja v kompleksni ravnini}
\vspace{0.5cm}

\begin{definicija}

Naj bo $D$ \textit{odprta} podmnožica kompleksne ravnine $\mathbb{C}$. \textit{Pot} v $D$ je \textit{odsekoma zvezno odvedljiva} preslikava $\gamma: [a, b] \rightarrow D$, kjer je $[a, b]$ \textit{zaprt} interval v $\mathbb{R}$. Množico $D$ imenujemo \textit{s potmi povezano}, če za vsaki točki $z, w \in D$ obstaja taka pot $\gamma$ v $D$, da je 
$$\gamma(a) ~=~ z ~~~\text{in}~~~ \gamma(b) ~=~ w.$$

\end{definicija}
\vspace{0.5cm}

\begin{opomba}
~
\begin{enumerate}

	\item Običajno pojem povezanosti s potmi definiramo brez zahteve, da mora biti preslikava $\gamma$ \textit{odsekoma odvedljiva}, zadošča le \textit{zveznost}, vendar se je lahko prepričati, da sta za \textit{odprte} množice $D$ v ravnini definiciji ekvivalentno. \footnote{Vsako zvezno preslikavo $\gamma: [a, b] \rightarrow D$ namreč lahko poljubno dobro aproksimiramo z \textit{odsekoma linearno} preslikavo, se pravi s poligonsko krivuljo. Še več, vsako tako pot je mogoče poljubno natančno aproksimirati z \textit{zvezno odveljivo} potjo, zato bi se v definiciji povezanosti s potmi za \textit{odprte} množice lahko omejili na \textit{zvezno odvedljive} poti.}
	
	\item Vsaka s potmi povezana množica $D$ je \textit{povezana} \footnote{Pomeni, da je iz enega kosa.} v naslednjem smislu: edini razcep množice $D$ na unijo $D = D_1 \cup D_2$ dveh \textit{disjunktnih odprtih (relativno v $D$)} podmnožic je, ko je ena od množic $D_i$ \textit{prazna}, druga pa enaka $D$. Za \textit{odprte} množice v ravnini sta pojma \textit{povezanosti} in \textit{povezanosti s potmi} ekvivalentna, torej $D$ je \textit{povzana} natanko takrat, ko je \textit{s potmi povezana}. 
	
\end{enumerate}

\end{opomba}
\vspace{0.5cm}

\begin{definicija}

\textit{Odprto povezano} množico v kompleksni ravnini $\mathbb{C}$ bomo imenovali \textit{območje}.

\end{definicija}
\vspace{0.5cm}

\begin{definicija}

\textit{Razširjena} kompleksna ravnina je $\hat{\mathbb{C}} := \mathbb{C} \cup \{\infty\}.$ Okolice točke $\infty$ so množice oblike $(\mathbb{C} \setminus K) \cup \{\infty\}$, kjer je $K$ \textit{kompaktna} podmnožica v $\mathbb{C}$.

\end{definicija}
\vspace{0.5cm}

% ************************************************************************************************* 

\subsection{Odvedljivost v kompleksnem smislu in konformnost}
\vspace{0.5cm}

% -------------------------------------------------------------------------------------------------

\subsubsection{Kompleksna odvedljivost}
\vspace{0.5cm}

\begin{definicija}

Naj bo $f: D \rightarrow \mathbb{C}$ funkcija, definirana na \textit{odprti} množici $D \subseteq \mathbb{C}$. Pravimo, da je $f$ \textit{odvedljiva v kompleksnem smislu} ali \textit{holomorfna} v točki $z \in D$, če obstaja limita
$$f'(z) ~:=~ \lim_{w \rightarrow z} \frac{f(w) - f(z)}{w - z}.$$
To limito imenujemo \textit{odvod funkcije $f$ v točki $z$}. Če je $f$ odvedljiva v vsaki točki $z \in D$, rečemo, da je $f$ \textit{holomorfna} na $D$

\end{definicija}
\vspace{0.5cm}

\begin{trditev}

Funkcija $f$, podana s potenčno vrsto 
$$f(z) ~=~ a_0 + a_1 z + a_2 z^2 + \ldots ~=~ \sum_{n=0}^{\infty} a_n z^n, ~~(a_n) \in \mathbb{C},$$
je \textit{holomorfna} na krogu $|z| < R$, kjer je njen odvod enak
$$f'(z) ~=~ g(z) ~:=~ \sum_{n=0}^{\infty} n a_n z^{n-1}.$$
Tukaj je $R$ konvergenčni polmer vrste $\mathlarger{\sum_{n=0}^{\infty} a_n z^n}$. 

\end{trditev}
\vspace{0.5cm}

\begin{posledica}

Če vrsta $f(z) = \mathlarger{\sum_{n=0}^{\infty} a_n (z - \alpha)^n}$ konvergira na krogu $|z - \alpha| < R$, potem je funkcija $f$ \textit{neskončnokrat odvedljiva (v kompleksnem smislu)} in
$$f^{(k)}(z) ~=~ \sum_{n=k}^{\infty} n(n-1) \ldots (n-k+1) a_n (z - \alpha)^{n-k}.$$
Torej je
$$a_k ~=~ \frac{f^{(k)}(\alpha)}{k!}.$$

\end{posledica}
\vspace{0.5cm}

% -------------------------------------------------------------------------------------------------

\subsubsection{Konformne preslikave}

\begin{definicija}

Naj bo $\gamma: [0,1] \rightarrow \mathbb{C}$ kaka \textit{zvezno odvedljiva} pot, ki naj gre skozi točko $z_0 \in \mathbb{C}$, torej $z_0 = \gamma(t_0)$ za kak $t_0 \in [0,1]$. Tangentni vektor v točki $z_0$ na pot $\gamma$ je tedaj (gledan kot kompleksno število) enak $\dot{\gamma}(t_0)$. Nadalje naj bo $f$ funkcija, ki je definirana in \textit{holomorfna} na kakem območju, ki vsebuje množico $[\gamma] := \gamma([0,1])$. Tangentni vektor v točki $f(z_0)$ na pot $f \circ \gamma$ je potem 
$$\frac{d}{dt}\left( f \circ \gamma \right)(t_0) ~=~ f'(z_0) \dot{\gamma}(t_0).$$
Pri tem privzamemo, da je $f'(z_0) \neq 0$. Dolžina tega tangentnega vektorja je torej $$|f'(z_0)||\dot{\gamma}(t_0)|.$$
njegov argument pa $\arg{\dot{\gamma}(t_0)} + \arg{f'(z_0)}$. \\

\noindent Za dve taki, v točki $z_0$ sekajoči poti $\gamma_1$ in $\gamma_2$, je kot med njima \footnote{Kot med njunima tangentnima vektorjema $\dot{\gamma_j}(t_j)$, kjer je $\gamma_j(t_j) = z_0$, $j = 1, 2$} enak 
$$\psi ~:=~ \arg{\dot{\gamma_2}(t_2)} - \arg{\dot{\gamma_1}(t_1)}.$$
Kot med potema $f \circ \gamma_1$ in $f \circ \gamma_2$ v presečišču $f(z_0)$ je torej
\begin{align*}
\arg{\frac{d}{dt}\left( f \circ \gamma_2 \right)(t_2)} - \arg{\frac{d}{dt}\left( f \circ \gamma_1 \right)(t_1)} ~=~ \\ 
~=~ (\arg{f'(z_0)} + \arg{\dot{\gamma_2}(t_2)}) - (\arg{f'(z_0)} + \arg{\dot{\gamma_1}(t_1)} ~&=~ \psi
\end{align*}
Torej preslikava $f$ ohranja kote med krivuljami. Zato imenujemo \textit{holomorfne} preslikave z neničelnim odvodom tudi $\textit{konformne preslikave}$.

\end{definicija}
\vspace{0.5cm}

% -------------------------------------------------------------------------------------------------

% *************************************************************************************************

\subsection{Cauchy-Riemannovi enakosti}
\vspace{0.5cm}

\begin{izrek}

Naj bosta $u(x, y)$ in $v(x, y)$ \textit{realni} funkciji na \textit{odprti} množici $D$ v ravnini. Če je funkcija $f = u + iv$ \textit{holomorfna} na $D$, sta $u$ in $v$ \textit{odvedljivi \footnote{V realnem smislu.}} in veljata Cauchy-Riemannovi enakosti
$$\frac{\partial u}{\partial x} ~=~ \frac{\partial v}{\partial y} ~~~\text{in}~~~ \frac{\partial v}{\partial x} ~=~ -\frac{\partial u}{\partial y}.$$ V obratno smer pa velja: če so partialni odvodi $\cfrac{\partial u}{\partial x}$, $\cfrac{\partial u}{\partial y}$, $\cfrac{\partial v}{\partial x}$ in $\cfrac{\partial v}{\partial y}$ \textit{zvezni} na $D$ in veljata Cauchy-Riemannovi enakosti, potem je $f$ \textit{holomorfna} na $D$ in
$$f' ~=~ \frac{\partial u}{\partial x} + i \frac{\partial v}{\partial x}.$$

\end{izrek}
\vspace{0.5cm}

\begin{definicija}

\textit{Dvakrat zvezno odvedljivo} funkcijo, definirano na odprti množici $D \subseteq \mathbb{R}^n$, imenujemo \textit{harmonična} na $D$, če je 
$$\Delta u ~:=~ \sum_{j=1}^n \frac{\partial^2 u}{\partial {x_j}^2}(u) ~=~ 0 ~~~(x \in D).$$
Operator $\Delta$ imenujemo \textit{Laplaceov operator}.

\end{definicija}
\vspace{0.5cm}

\begin{trditev}

Realni in imaginarni del \textit{holomorfne} funkcije sta \textit{harmonični} funkciji. Na \textit{enostavno povezanem} območju je vsaka harmonična funkcija realni del kake holomorfne funkcije.

\end{trditev}
\vspace{0.5cm}

% *************************************************************************************************

\subsection{Integriranje kompleksnih funkcij}
\vspace{0.5cm}

\begin{definicija}

\textit{Integral kompleksne funkcije} $f = u + iv$, kjer sta $u$ in $v$ \textit{realni integrabilni} funkciji na intervalu $[a, b]$, definiramo kot
$$\int_a^b f(t) dt ~=~ \int_a^b u(t) dt + i\int_a^b v(t) dt.$$
Funkcijo $f = u + iv$ imenujemo \textit{integrabilna}, če sta integrabilni funkciji $u$ in $v$. \\

\noindent Integral kompleksnih funkcij ima podobne lastnosti kot integral realnih funkcij, npr.
$$\int_a^b \left( \alpha f + \beta g \right) dt ~=~ \alpha \int_a^b f dt + \beta \int_a^b g dt$$
za poljubni konstanti $\alpha, \beta \in \mathbb{C}$ in \textit{integrabilni} kompleksni funkciji $f$ in $g$. Integral kompleksne funkcija $f$ na intervalu $[a, b]$ bi lahko ekvivalentno definirali tudi kot limito Riemannovih vsot 
$$\sum_{j=1}^n f(\xi_j)(t_j - t_{j-1}), ~~~\xi_j \in [t_{j-1}, t_j],$$
ko gre širina $\mathlarger{\max_{1 \leq j \leq n}(t_j - t_{j-1})}$ delitve 
$$a = t_1 < t_1 < \ldots < t_{j-1} < t_j < \ldots < t_n = b$$
 intervala $[a, b]$ proti $0$.

\end{definicija}
\vspace{0.5cm}

\begin{trditev}

$$\left| \int_a^b f dt \right| ~\leq~ \int_a^b |f| dt, ~~~(a \leq b).$$

\end{trditev}
\vspace{0.5cm}

\begin{definicija}

Če je $\gamma: [a, b] \rightarrow D$ \textit{zvezno odvedljiva}, $f: D \rightarrow \mathbb{C}$ pa \textit{zvezna} funkcija, kjer je $D \subseteq \mathbb{C}$ poljubna množica, ki vsebuje $\gamma([a, b])$, definiramo \textit{krivuljni integral} kot
$$\int_{\gamma} f(z) dz ~:=~ \int_a^b f(\gamma(t)) \gamma'(t) dt.$$
Če je $\gamma: [a, b] \rightarrow D$ poljubna pot \footnote{Torej le \textit{odsekoma zvezno odvedljiva}}, pa najprej $[a, b]$ razdelimo na podintervale $[t_{j-1}, t_j]$, na katerih je $\gamma$ \textit{zvezno odvedljiva}, in definiramo $\mathlarger{\int_{\gamma} f(z) dz}$ kot vsoto integralov $\mathlarger{\int_{t_{j-1}}^{t_j} f(\gamma(t))\gamma'(t) dt}$.

\end{definicija}
\vspace{0.5cm}

\begin{trditev}

Če je $\varphi: [c, d] \rightarrow [a, b]$ \textit{naraščujoča zvezno odvedljiva bijekcija}, je
$$\int_{\gamma \circ \varphi} f(z) dz ~=~ \int_{\gamma} f(z) dz. \footnote{Krivuljni integral je torej neodvisen od parametrizacije krivulje.}$$

\end{trditev}
\vspace{0.5cm}

\begin{izrek}

$$\left| \int_{\gamma} f(z) dz \right|  ~\leq~ \int_{\gamma} |f(z)| dz.$$

\end{izrek}
\vspace{0.5cm}

\begin{definicija}

Poti $\gamma: [a, b] \rightarrow D$ \textit{nasprotna} je pot
$$\gamma^-: [a, b] \rightarrow D, ~~~ \gamma^- ~=~ \gamma(a + b - t).$$
Ko teče $t$ od $a$ do $b$, potuje točja $\gamma(t)$ od $\gamma(a)$ do $\gamma(b)$, točka $\gamma^-(t)$ pa v nasprotni smeri od $\gamma(b)$ do $\gamma(a)$.

\end{definicija}
\vspace{0.5cm}

\begin{trditev}

$$\int_{\gamma^-} f(z) dz ~=~ -\int_{\gamma} f(z) dz.$$

\end{trditev}
\vspace{0.5cm}

\begin{trditev}

$$\int_{\gamma_1 + \gamma_2} f(z) dz ~=~ \int_{\gamma_1} f(z) dz + \int_{\gamma_2} f(z) dz.$$

\end{trditev}
\vspace{0.5cm}

\begin{definicija}

Pot $\gamma: [a, b] \rightarrow D$ je \textit{sklenjena}, če je $\gamma(b) = \gamma(a)$. Če imata poti $\gamma_1$ in $\gamma_2$ isto začetno in isto končno točko, potem je $\gamma_1 + {\gamma_2}^-$ sklenjena pot.

\end{definicija}
\vspace{0.5cm}

\begin{lema}

Naj bo $f: D \rightarrow \mathbb{C}$ \textit{(zvezna)} funkcija. Potem je
$$\int_{\gamma_1} f(z) dz ~=~ \int_{\gamma_2} f(z) dz$$
za vsaki dve poti $\gamma_1$ in $\gamma_2$ v $D$, ki imata isto začetno in isto končno točko, natanko tedaj, ko je 
$$\int_{\gamma} f(z) dz ~=~ 0$$
za vsako \textit{sklenjeno} pot $\gamma$ v $D$.

\end{lema}
\vspace{0.5cm}

\begin{lema}

Za vsako \textit{holomorfno} funkcijo $F: D \rightarrow \mathbb{C}$ z zveznim odvodom in vsako pot $\gamma: [a, b] \rightarrow D$ je 
$$\int_{\gamma} F'(z) dt ~=~ F(\gamma(b)) - F(\gamma(a)).\footnote{Privzamemo, da je $F'$ zvezna funkcija.}$$

\end{lema}
\vspace{0.5cm}

\begin{izrek}

Naj bo $f: D \rightarrow \mathbb{C}$ taka \textit{zvezna} funkcija na območju $D$, da je 
$$\int_{\gamma} f(z) dz ~=~ 0$$
za vsako sklenjeno pot $\gamma$ v $D$. Potem obstaja taka \textit{holomorfna} funkcija $F$ na $D$, da je 
$$f = F'.$$

\end{izrek}
\vspace{0.5cm}

% *************************************************************************************************

\subsection{Ovojno število}
\vspace{0.5cm}

\begin{definicija}

Za sklenjeno pot $\gamma$ v $\mathbb{C}$ in vsako točko $\alpha$, ki ne leži na sliki poti $\gamma$, je \textit{ovojno število} (ali \textit{indeks}) definirano kot
$$I_{\gamma}(\alpha) ~=~ \frac{1}{2 \pi i} \int_{\gamma} \frac{dz}{z - \alpha}.$$

\end{definicija}
\vspace{0.5cm}

% *************************************************************************************************

\subsection{Cauchyjeva-Greenova formula}
\vspace{0.5cm}

\begin{izrek}[Cauchyjev izrek]

Naj bo $\gamma$ taka \textit{sklenjena} pot v \textit{neprazni odprti} množici $D \subseteq \mathbb{C}$, da je $I_{\gamma}(w) = 0$ za $\forall w \in \mathbb{C} \setminus D$. Potem za vsako \textit{v kompleksne smislu odvedljivo} funkcijo $f: D \rightarrow \mathbb{C}$ velja
$$\int_{\gamma} f(z) dz ~=~ 0.$$

\end{izrek}
\vspace{0.5cm}

\begin{izrek}[Cauchyjeva formula]

Naj bo $\gamma$ taka \textit{sklenjena} pot v \textit{neprazni odprti} množici $D \subseteq \mathbb{C}$, da je $I_{\gamma}(\zeta) = 0$ za $\forall \zeta \in \mathbb{C} \setminus D$. Potem za vsako \textit{holomorfno} funkcijo $f: D \rightarrow \mathbb{C}$ velja
$$I_{\gamma}(w)f(w) ~=~ \frac{1}{2 \pi i} \int_{\gamma} \frac{f(z)}{z - w} dw$$
za $\forall w \in D \setminus [\gamma]$.

\end{izrek}
\vspace{0.5cm}

\begin{posledica}[Cauchyjeva formula za kolobar]

Naj bo $f$ \textit{holomorfna} funkcija na kaki okolici kolobarja $K := \{ z \in \mathbb{C}: ~r \leq |z - \alpha| \leq R \}$ \footnote{Kjer sta $r < R$ \textit{nenegativni} konstanti.}. Potem za vsak $w$ iz notranjosti kolobarja $K$ \footnote{Torej $r < |w - \alpha| < R$} velja
$$f(w) ~=~ \frac{1}{2 \pi i} \oint_{|z - \alpha| = R} \frac{f(z)}{z - w} dz - \frac{1}{2 \pi i} \oint_{|z - \alpha| = r} \frac{f(z)}{z - \alpha} dz,$$
kjer sta oba integrala po \textit{pozitivno orientiranih} krožnicah. \footnote{Kolobar leži potem na desni strani notranje krožnice, od tod predznak $-$ pred drugim integralom.}

\end{posledica}
\vspace{0.5cm}

\begin{posledica}

Če je torej  $f$ holomorfna na kaki okolici kroga $|z - \alpha| \leq R$, potem je za vsak $w$ v notranjosti tega kroga
$$f(w) ~=~ \frac{1}{2 \pi i} \oint_{|z - \alpha| = R} \frac{f(z)}{z - w} dz.$$

\end{posledica}
\vspace{0.5cm}

\begin{posledica}

\textit{Holomorfna} funkcija na območju $D$ je \textit{neskončnokrat odvedljiva}. Za $\forall n \in \mathbb{N}$ in vsak \textit{zaprt} krog $\overline{D}(\alpha, r)$, vsebovan v $D$, velja za $\forall w \in D(\alpha, r)$ enakost
$$f^{(n)}(w) ~=~ \frac{n!}{2 \pi i} \oint_{|z - \alpha| = r} \frac{f(z)}{(z - w)^{n+1}} dz.$$
Če je torej $|f|$ \textit{omejena}, se pravi $|f(z)| \leq M$ za kak $M$ in $\forall z \in D$, potem velja \textit{Cauchyjeva ocena za odvod}
$$|f^{(n)}(\alpha)| ~\leq~ \frac{Mn!}{r^n}.$$ 

\end{posledica}
\vspace{0.5cm}

\begin{posledica}[Liouvillow izrek]

\textit{Omejena holomorfna} funkcija $f$ na $\mathbb{C}$ je \\konstantna.

\end{posledica}
\vspace{0.5cm}

\begin{posledica}[Osnovni izrek algebre]

Vsak \textit{nekonstanten kompleksen} polinom 
$$p(z) ~=~ z^n + a_{n_1}z^{n-1} + \ldots + a_0, ~~~(n \geq 1)$$
ima vsaj eno kompleksno ničlo.

\end{posledica}
\vspace{0.5cm}

% *************************************************************************************************

\subsection{Razvoj v Laurentovo in v Taylorjevo vrsto}
\vspace{0.5cm}

\begin{izrek}[Laurentov razvoj]

Funkcijo $f$, \textit{holomorfno} na kaki okolici \\kolobarja $K = \{ z \in \mathbb{C}: ~r \leq |z - \alpha| \leq R \}$, lahko za vsak $w$ iz notranjosti kolobarja razvijemo v \textit{Laurentovo vrsto}
$$f(w) ~=~ \sum_{n=0}^{\infty} c_n (w - \alpha)^n + \sum_{n=-1}^{-\infty} c_n (w - \alpha)^n,$$
kjer je 
$$c_n ~=~ \frac{1}{2 \pi i} \int_{|z - \alpha| = R} \frac{f(z)}{(z - \alpha)^{n+1}} dz ~~~\textit{za}~ n \geq 0$$
in
$$c_n ~=~ \frac{1}{2 \pi i} \int_{|z - \alpha| = r} \frac{f(z)}{(z - \alpha)^{n+1}} dz ~~~\textit{za}~ n \leq -1.$$
Prva vrsta v $f(w)$ \footnote{Regularni del Laurentovega razvoja.} konvergira za vse $w$ znotraj kroga $|w - \alpha| < R$, druga \footnote{Glavni del Laurentovega razvoja.} pa ua vse $w$ zunaj kroga $|w - \alpha| \leq r$. Vrsti predstavljata \textit{holomorfni} funkciji na območjih konvergence.

\end{izrek}
\vspace{0.5cm}

\begin{posledica}[Taylorjev razvoj]

Funkcijo $f$, ki je \textit{holomorfna} na okolici \textit{zaprtega} kroga $\overline{D}(\alpha, R)$, lahko razvijemo v \textit{Taylorjevo vrsto}
$$f(w) ~=~ \sum_{n=0}^{\infty} c_n (w - \alpha)^n ~~~(|w - \alpha| < R),$$
kjer je 
$$c_n ~=~ \frac{1}{2 \pi i} \int_{|z - \alpha| = R} \frac{f(z)}{(z - \alpha)^{n+1}} dz$$
in kjer integriramo krožnici po pozitivni smeri.

\end{posledica}
\vspace{0.5cm}

\begin{trditev}

Ničle \textit{holomorfne} funckije $f: D \rightarrow \mathbb{C}$ na območju $D$ nimajo stekališč v $D$, če $f$ ni identično enaka $0$.

\end{trditev}
\vspace{0.5cm}

\begin{posledica}

Če se \textit{holomorfni} funkciji $f$ in $g$, definirani na območju $D$, ujemata na kaki podmnožici $S \subseteq D$, ki ima kako stekališče v $D$, potem je $g ~=~ f$.  

\end{posledica}
\vspace{0.5cm}

\begin{trditev}

Naj bo $\alpha$ \textit{izolirana singularna} točka \textit{holomorfne} funkcije $f$. Če je $f$ \textit{omejena} v kaki okolici točke $\alpha$, potem je $\alpha$ \textit{premostljiva singularna} točka. \footnote{Glej opombo v skripti.}

\end{trditev}
\vspace{0.5cm}

\begin{izrek}[Casorati-Weierstrass]

Če je $\alpha$ \textit{bistvena singularna} točka funkcije $f$ \footnote{Ta je sicer \textit{holomorfna} na kaki okolici te točke, razen v $\alpha$}, potem za $\forall w \in \mathbb{C}$ ter $\forall \varepsilon > 0$ in $\forall \delta > 0$ obstaka tak $z \in D(\alpha, \delta) \setminus \{ \alpha \}$, da je 
$$|f(z) - w| ~<~ \varepsilon.$$ 

\end{izrek}
\vspace{0.5cm}

% *************************************************************************************************

\pagebreak

% #################################################################################################

\end{document}